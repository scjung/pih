\chapter{\Large{Recursive functions}}

%%%%%%%%%%%%%%%%%%%%%%%%%%%%%%%%%%%%%%%%%%%%%%%%%%%%%%%%%%%%%%%%%%%%%%
\exercise{6.1}
\haskell{./src/ch06-ex01.hs}

%%%%%%%%%%%%%%%%%%%%%%%%%%%%%%%%%%%%%%%%%%%%%%%%%%%%%%%%%%%%%%%%%%%%%%
\exercise{6.2}
\begin{itemize}
\item $length~[1,2,3]$
  \begin{align*}
      & length~[1,2,3] \\
    = & \qquad \{~ \text{applying}~length ~\} \\
      & 1 + length~[2,3] \\
    = & \qquad \{~ \text{applying}~length ~\} \\
      & 1 + (1 + length~[3]) \\
    = & \qquad \{~ \text{applying}~length ~\} \\
      & 1 + (1 + (1 + length~[])) \\
    = & \qquad \{~ \text{applying}~length ~\} \\
      & 1 + (1 + (1 + 0)) \\
    = & \qquad \{~ \text{applying}~+ ~\} \\
      & 1 + (1 + 1) \\
    = & \qquad \{~ \text{applying}~+ ~\} \\
      & 1 + 2 \\
    = & \qquad \{~ \text{applying}~+ ~\} \\
      & 3
  \end{align*}
\item $drop~3~[1,2,3,4,5]$
  \begin{align*}
      & drop~3~[1,2,3,4,5] \\
    = & \qquad \{~ \text{applying}~drop ~\} \\
      & drop~2~[2,3,4,5] \\
    = & \qquad \{~ \text{applying}~drop ~\} \\
      & drop~1~[3,4,5] \\
    = & \qquad \{~ \text{applying}~drop ~\} \\
      & drop~0~[4,5] \\
    = & \qquad \{~ \text{applying}~drop ~\} \\
      & [4,5]
  \end{align*}
\item $init~[1,2,3]$
  \begin{align*}
      & init~[1,2,3] \\
    = & \qquad \{~ \text{applying}~init ~\} \\
      & 1:init~[2,3] \\
    = & \qquad \{~ \text{applying}~init ~\} \\
      & 1:(2:init~[3]) \\
    = & \qquad \{~ \text{applying}~init ~\} \\
      & 1:(2:[]) \\
    = & \qquad \{~ \text{applying}~: ~\} \\
      & 1:[2] \\
    = & \qquad \{~ \text{applying}~: ~\} \\
      & [1,2]
  \end{align*}
\end{itemize}

%%%%%%%%%%%%%%%%%%%%%%%%%%%%%%%%%%%%%%%%%%%%%%%%%%%%%%%%%%%%%%%%%%%%%%
\exercise{6.3}
다음은 구현된 코드이다. 대부분 \texttt{Prelude}에 정의된 함수와 이름 충돌이
발생하는데, 함수 이름 끝에 따옴표를 붙이거나 \texttt{Main} 모듈을 명시하여
문제를 해결하였다.

연산자 \texttt{!!}의 경우에는 주어진 수가 주어진 리스트의 인덱스 범위를
벗어나는 경우가 처리되지 않았다. 이러한 경우에는 적절한 예외를 발생시켜야
하는데, 아직 예외에 대해서는 다루지 않았으므로 여기서는 무시하고 지나가도록
한다. 사실 실제로 인덱스 범위를 벗어나는 경우를 시험해보면 (의미는 정확하지
않더라도) 처리되지 않은 패턴이 있다는 예외가 발생한다.
\haskell{./src/ch06-ex03.hs}

%%%%%%%%%%%%%%%%%%%%%%%%%%%%%%%%%%%%%%%%%%%%%%%%%%%%%%%%%%%%%%%%%%%%%%
\exercise{6.4}
\haskell{./src/ch06-ex04.hs}

%%%%%%%%%%%%%%%%%%%%%%%%%%%%%%%%%%%%%%%%%%%%%%%%%%%%%%%%%%%%%%%%%%%%%%
\exercise{6.5}
\haskell[7]{./src/ch06-ex05.hs}

%%%%%%%%%%%%%%%%%%%%%%%%%%%%%%%%%%%%%%%%%%%%%%%%%%%%%%%%%%%%%%%%%%%%%%
\exercise{6.6}
\begin{itemize}
\item $sum$
  \begin{enumerate}
  \item 타입 정의하기: 일단 단순히 정수의 리스트를 받아서 원소의 합을 내주는
    함수에서 시작하자. 함수의 타입은 다음과 같다.
    \[sum~::~[Int]~\rightarrow~Int\]
  \item 경우를 나열하기: 리스트를 받으므로, 가능한 경우는 빈 리스트인 경우와
    하나 이상의 원소가 있는 경우이다.
    \[\begin{array}{lcl}
      sum~[]       & = & \\
      sum~(x : xs) & = &
    \end{array}\]
  \item 간단한 경우를 정의하기: 우선 빈 리스트가 주어지는 경우를 고려하자. 이
    때의 합은 $0$으로 한다.
    \[\begin{array}{lcl}
      sum~[]       & = & 0 \\
      sum~(x : xs) & = &
    \end{array}\]
  \item 나머지 경우를 정의하기: 원소가 하나 이상인 경우에는, 재귀적으로
   정의한다. 뒷 리스트의 합을 함수 $sum$을 사용하여 재귀적으로 구하고, 이를 맨
   앞의 수와 더하면 된다.
    \[\begin{array}{lcl}
      sum~[]       & = & 0 \\
      sum~(x : xs) & = & x + sum~xs
    \end{array}\]
  \item 일반화 및 단순화 작업: 앞의 타입 정의에서는 주어진 리스트의 원소가
    $Int$ 타입이라고 하였으나, 사실 `더할' 수 있는 모든 것은 원소가 될 수
    있다. 그리고 어떤 타입의 값을 더하 결과 역시 이전 타입과 같다. 결과적으로
    다음과 같이 리스트 원소 및 결과 타입을 $Num$에 해당하는 타입으로 일반화
    시킬 수 있다.
    \[\begin{array}{lcl}
      sum          & :: & Num~a \Rightarrow a \rightarrow a \\
      sum~[]       &  = & 0 \\
      sum~(x : xs) &  = & x + sum~xs
    \end{array}\]
  \end{enumerate}

\item $take$
  \begin{enumerate}
  \item 타입 정의하기: 가져올 원소의 갯수에 해당하는 정수와 리스트를 받아서
    가져온 원소가 들어있는 리스트를 내준다. 주어진 리스트와 내놓을
    리스트의 타입은 같아야 한다. 이를 타입으로 정리하면 다음과 같다.
    \[take~::~Int~\rightarrow~[a]~\rightarrow~[a]\]
  \item 경우를 나열하기: 정수와 리스트를 받으므로, 가능한 모든 경우는 정수가
    $0$,$n+1$인 경우와 리스트가 빈 리스트, 하나 이상의 원소가 있는 경우
    각각을 짝지은 것이다. 정리하면 다음과 같다.
    \[\begin{array}{lcl}
      take~0~[]           & = & \\
      take~0~(x : xs)     & = & \\
      take~(n+1)~[]       & = & \\
      take~(n+1)~(x : xs) & = &
    \end{array}\]
  \item 간단한 경우를 정의하기: 주어진 정수가 0인 경우에는 아무것도 가져오지
    않으므로 결과는 빈 리스트이다. 또한 정수가 0보다 크고 리스트가 빈 리스트인
    경우에는 더 이상 가져올 것이 없으므로 이 경우 역시 빈 리스트이다.
    \[\begin{array}{lcl}
      take~0~[]           & = & [] \\
      take~0~(x : xs)     & = & [] \\
      take~(n+1)~[]       & = & [] \\
      take~(n+1)~(x : xs) & = &
    \end{array}\]
  \item 나머지 경우를 정의하기: 정수가 0보다 크고 원소가 하나 이상인 리스트가
    주어진 경우에는 함수 $take$를 재귀적으로 사용하여, 리스트 꼬리에서 원소를
    주어진 정수보다 하나 덜 가져온 후, 만들어진 리스트 맨 앞에 주어진 리스트의
    머리 원소를 붙이면 된다.
    \[\begin{array}{lcl}
      take~0~[]           & = & [] \\
      take~0~(x : xs)     & = & [] \\
      take~(n+1)~[]       & = & [] \\
      take~(n+1)~(x : xs) & = & x : take~n~xs
    \end{array}\]
  \item 일반화 및 단순화 작업: $Int$ 뿐만 아니라 어떠한 정수라도 리스트에서
    가져올 원소의 갯수를 나타내는 인자로 쓰일 수 있으므로, 이 인자의 타입을
    $Integral$에 해당하는 타입으로 더 일반화 시킬 수 있다.
    \[\begin{array}{lcl}
      take & :: & Integral~b~\Rightarrow~b~\rightarrow~[a]~\rightarrow~[a]\\
      take~0~[]           & = & [] \\
      take~0~(x : xs)     & = & [] \\
      take~(n+1)~[]       & = & [] \\
      take~(n+1)~(x : xs) & = & x : take~n~xs
    \end{array}\]
    또한 첫 인자가 0일 경우에는 어떠한 리스트가 오더라도 결과가 같으므로
    아래와 같이 경우의 수를 줄일 수 있다.
    \[\begin{array}{lcl}
      take & :: & Integral~b~\Rightarrow~b~\rightarrow~[a]~\rightarrow~[a]\\
      take~0~\_           & = & [] \\
      take~(n+1)~[]       & = & [] \\
      take~(n+1)~(x : xs) & = & x : take~n~xs
    \end{array}\]
  \end{enumerate}

\item $last$
  \begin{enumerate}
  \item 타입 정의하기: 리스트를 받고 리스트의 마지막 원소를 내준다. 따라서
    타입은 다음과 같다.
    \[last~::~[a]~\rightarrow~a\]
  \item 경우를 나열하기: 리스트를 받으므로, 가능한 경우는 빈 리스트인 경우와
    하나 이상의 원소가 있는 경우이다. 그런데 빈 리스트는 가능한 입력이 아니므로
    코드에서 제외한다.
    \[\begin{array}{lcl}
      last~(x : xs) & = &
    \end{array}\]
  \item 간단한 경우를 정의하기: 제시된 경우는 다시 꼬리 리스트가 빈 리스트인
    경우와 아닌 경우로 나눌 수 있다. 전자의 경우에는 맨 앞 원소 $x$가 마지막
    원소이다.
    \[\begin{array}{lclcl}
      last~(x : xs) &|& null~xs   & = & x\\
                    &|& otherwise & = & \\
    \end{array}\]
  \item 나머지 경우를 정의하기: 꼬리 리스트가 빈 리스트가 아닌 경우에는 꼬리
    리스트에서 마지막 원소를 재귀적으로 찾으면 된다.
    \[\begin{array}{lclcl}
      last~(x : xs) &|&null~xs   & = & x\\
                    &|&otherwise & = & last~xs\\
    \end{array}\]
  \item 일반화 및 단순화 작업: 앞에서 쓰인 가드를 패턴 매칭으로 바꾸어
    정리하면 다음과 같다.
    \[\begin{array}{lcl}
      last & :: & [a]~\rightarrow~a \\
      last~[x]      & = & x \\
      last~(\_ : xs) & = & last~xs \\
    \end{array}\]
  \end{enumerate}
\end{itemize}


%%% Local Variables: 
%%% mode: latex
%%% TeX-master: "master"
%%% End: 