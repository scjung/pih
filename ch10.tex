\chapter{\Large{Declaring types and classes}}

%%%%%%%%%%%%%%%%%%%%%%%%%%%%%%%%%%%%%%%%%%%%%%%%%%%%%%%%%%%%%%%%%%%%%%
\exercise{10.1}
\haskell[20]{./src/ch10-ex01.hs}

%%%%%%%%%%%%%%%%%%%%%%%%%%%%%%%%%%%%%%%%%%%%%%%%%%%%%%%%%%%%%%%%%%%%%%
\exercise{10.2}
\haskell[3]{./src/ch10-ex02.hs}

이전 구현은 주어진 트리가 노드인 경우, 가능한 경우마다 따로 값 비교를
수행한다. 따라서 최악의 경우에는 주어진 수가 노드의 수보다 크다는 사실을 알기
위해 같음, 작음, 큼을 세 번 비교하게 된다. 하지만 여기서 새로 작성한
코드는 대소여부를 판별한 결과가 세 경우 각각을 나타내므로 $compare$를
여러번 사용할 필요가 없어, 더 효율적이다.

%%%%%%%%%%%%%%%%%%%%%%%%%%%%%%%%%%%%%%%%%%%%%%%%%%%%%%%%%%%%%%%%%%%%%%
\exercise{10.3}
\haskell[5]{./src/ch10-ex03.hs}

%%%%%%%%%%%%%%%%%%%%%%%%%%%%%%%%%%%%%%%%%%%%%%%%%%%%%%%%%%%%%%%%%%%%%%
\exercise{10.4}
\haskell[5]{./src/ch10-ex04.hs}

%%%%%%%%%%%%%%%%%%%%%%%%%%%%%%%%%%%%%%%%%%%%%%%%%%%%%%%%%%%%%%%%%%%%%%
\exercise{10.5}
우선 다음과 같이 새 요소를 데이터 선언에 추가한다.
\haskellpart{./src/ch10-ex05.hs}{6}{12}

이에 따라 $eval$ 함수에 계산 코드를 추가한다.
\haskellpart{./src/ch10-ex05.hs}{22}{23}

마지막으로 $vars$ 함수에 변수를 찾는 코드를 추가한다.
\haskellpart{./src/ch10-ex05.hs}{31}{32}

%%%%%%%%%%%%%%%%%%%%%%%%%%%%%%%%%%%%%%%%%%%%%%%%%%%%%%%%%%%%%%%%%%%%%%
\exercise{10.6}

먼저 작성할 프로그램이 받는 입력의 문법을 정의해야 한다. 우선 각
수학 기호에 해당하는 실제 입력 문자열 토큰을 정의해보자.

\begin{center}
  \begin{tabular}{cc}
    수학 기호          & 입력 문자열 \\
    \hline
    $T$               & \texttt{T} \\
    $F$               & \texttt{F} \\
    $\neg$            & \texttt{!} \\
    $\wedge$          & \texttt{/\textbackslash} \\
    $\vee$            & \texttt{\textbackslash/} \\
    $\Rightarrow$     & \texttt{=>} \\
    $\Leftrightarrow$ & \texttt{<=>} \\
  \end{tabular}
\end{center}

다음은 정의에 따라 작성된 요소 각각을 파싱하는 함수이다. 8장에서 사용된 코드를
재사용하고 있음에 유의하라.
\haskellpart{./src/ch10-ex06.hs}{126}{145}

이제 이러한 요소로 어떻게 명제가 구성되는지를 나타내는 문법을 정의하자. 문법을
정의할 때는 모호한 경우가 발생하지 않도록 우선순위, 결합방향을 주의깊게
고려해야 한다. 결합방향은 모든 경우에 대해 우결합으로 하고\footnote{즉,
  `\texttt{a => b => c}'는 `\texttt{a => (b => c)}'와 같다.}, 우선순위는 제일
높은 순으로 $\neg \rightarrow \wedge, \vee \rightarrow \Rightarrow,
\Leftrightarrow$의 순서를 따른다. 정의된 문법은 다음과 같다.

\[\begin{array}{lcl}
   prop & ::= & term ~\texttt{=>}~ prop ~|~ term ~\texttt{<=>}~ prop ~|~ term \\
   term & ::= & not ~\texttt{/\textbackslash} ~term ~|~ not ~\texttt{\textbackslash/} ~term ~|~ not \\
    not & ::= & \texttt{!} ~not ~|~ paren \\
  paren & ::= & \texttt{(} ~prop~ \texttt{)} ~|~ const ~|~ var \\
  const & ::= & \texttt{T} ~|~ \texttt{F} \\
    var & ::= & \texttt{a} ~|~ \texttt{b} ~|~ \cdots
\end{array}\]

변수명은 소문자로 이루어진다.\footnote{대문자로도 이루어지게 하면, 참과 거짓을
  뜻하는 \texttt{T}와 \texttt{F}가 변수로도 해석될 수 있다. 여기서는 이런
  모호한 경우를 피하기 위해 변수는 반드시 소문자로만 쓰이도록 하였다.} 이제
정의한 문법에 따라 파서 코드를 작성하면 다음과 같다.

\haskellpart{./src/ch10-ex06.hs}{147}{189}

다음은 사용자의 입력을 받아 토톨로지 여부를 출력해주는 사용자 인터페이스
코드이다.

\haskell[191]{./src/ch10-ex06.hs}

다음은 교재에서 제시된 명제가 항진식인지를 작성된 프로그램을 통해 확인해 본
모습이다.

\begin{lstlisting}
*Main> taut
tautology? a /\ !a
no
tautology? (a /\ b) => a
yes
tautology? a => (a /\ b)
no
tautology? (a /\ (a => b)) => b
yes
\end{lstlisting}

%%%%%%%%%%%%%%%%%%%%%%%%%%%%%%%%%%%%%%%%%%%%%%%%%%%%%%%%%%%%%%%%%%%%%%
\exercise{10.7}
\haskell{./src/ch10-ex07.hs}

%%%%%%%%%%%%%%%%%%%%%%%%%%%%%%%%%%%%%%%%%%%%%%%%%%%%%%%%%%%%%%%%%%%%%%
\exercise{10.8}
\haskell{./src/ch10-ex08.hs}


%%% Local Variables: 
%%% mode: latex
%%% TeX-master: "master"
%%% End: 
      