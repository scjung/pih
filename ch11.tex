\chapter{\Large{The countdown problem}}

%%%%%%%%%%%%%%%%%%%%%%%%%%%%%%%%%%%%%%%%%%%%%%%%%%%%%%%%%%%%%%%%%%%%%%
\exercise{11.1}
\haskellpart{./src/ch11-ex01.hs}{17}{18}

%%%%%%%%%%%%%%%%%%%%%%%%%%%%%%%%%%%%%%%%%%%%%%%%%%%%%%%%%%%%%%%%%%%%%%
\exercise{11.2}
\haskell{./src/ch11-ex02.hs}

%%%%%%%%%%%%%%%%%%%%%%%%%%%%%%%%%%%%%%%%%%%%%%%%%%%%%%%%%%%%%%%%%%%%%%
\exercise{11.3}
만일 $split$을 다음과 같이 빈 리스트가 쌍에 포함되도록 수정한다면,
\begin{lstlisting}
split         :: [a] -> [([a], [a])]
split []       = []
split [x]      = [([x], [])]
split (x : xs) = ([], (x : xs)) : ([x], xs) : [(x : ls, rs) | (ls, rs) <- split xs]
\end{lstlisting}

\noindent
결과에는 쌍의 왼쪽 혹은 오른쪽 원소가 나누려는 리스트 전체인 쌍이 포함되게
된다. 따라서 $results$ 함수의 리스트 조건식 내 생성자에서 쓰이는 재귀호출 중
하나는 원래의 리스트에 대해 다시 호출되어, 이에 따라 재귀호출이 무한히 일어나게
된다. 따라서 $solutions$ 계산은 끝나지 않는다.

%%%%%%%%%%%%%%%%%%%%%%%%%%%%%%%%%%%%%%%%%%%%%%%%%%%%%%%%%%%%%%%%%%%%%%
\exercise{11.4}
구성할 수 있는 모든 계산식의 수는 다음과 같이 구할 수 있다.
\begin{lstlisting}
*Main> length [e | ns' <- choices [1,3,7,10,25,50], e <- exprs ns']
33665406
\end{lstlisting}

이 중 실제로 계산할 수 있는 계산식의 수는 다음과 같이 구할 수 있다.
\begin{lstlisting}
*Main> length [r | ns' <- choices [1,3,7,10,25,50], e <- exprs ns', r <- eval e]
4672540
\end{lstlisting}

%%%%%%%%%%%%%%%%%%%%%%%%%%%%%%%%%%%%%%%%%%%%%%%%%%%%%%%%%%%%%%%%%%%%%%
\exercise{11.5}
다음과 같이 $valid$ 함수를 수정한다. 이 때 정수만이 취급되어야 하므로 $Div$의
경우에서 나누어 떨어지는지 여부를 검사하는 것은 지우지 말아야 한다. 또한
$0$으로 나누는 경우가 발생하지 않도록 추가적으로 검사를 수행해야 한다.

\begin{lstlisting}[language=Haskell]
valid        :: Op -> Int -> Int -> Bool
valid Add _ _ = True
valid Sub _ _ = True
valid Mul _ _ = True
valid Div x y = y /= 0 && x `mod` y == 0
\end{lstlisting}

다음과 같이 실제로 계산할 수 있는 계산식의 수를 구할 수 있다.
\begin{lstlisting}
*Main> length [r | ns <- choices [1,3,7,10,25,50], e <- exprs ns, r <- eval e]
10839369
\end{lstlisting}


%%%%%%%%%%%%%%%%%%%%%%%%%%%%%%%%%%%%%%%%%%%%%%%%%%%%%%%%%%%%%%%%%%%%%%
\exercise{11.6}

\begin{enumerate}
\item 제곱을 사용한 계산식을 허용하기 위해, 우선 연산자 정의를 다음과
  같이 수정한다.
  \haskellpart{./src/ch11-ex06.hs}{4}{4}
  이에 따라 관련된 함수인 $valid$, $apply$에 제곱에 관한 경우를 추가해준다.
  \haskellpart{./src/ch11-ex06.hs}{7}{19}
  마지막으로 해답을 구할 때 쓰이는 $ops$ 리스트에 제곱 연산자를 추가한다.
  \haskellpart{./src/ch11-ex06.hs}{67}{68}
  다음은 수정한 프로그램을 사용해 $1,4,6,20$에서 $100$을 만들 수 있는
  계산식을 알아본 것이다. 여기서 쓰인 $putSolutions$ 함수는 찾아낸 $Expr$
  리스트를 보기 쉽게 출력해주는 $Expr \rightarrow IO~()$ 타입의 함수이다.
  \begin{lstlisting}
*Main> putSolutions (solutions [1,4,6,20] 100)
(6 - 1) * 20 = 100
(1 + 4) * 20 = 100
4 * ((6 - 1) + 20) = 100
4 * (6 + (20 - 1)) = 100
4 * ((6 + 20) - 1) = 100
((1 ^ 6) + 4) * 20 = 100
(6 - (1 ^ 4)) * 20 = 100
  \end{lstlisting}
\item 다음은 정확한 해답을 찾지 못할 때 근접한 해답을 대신 내주는 함수
  $solutionsOrNearests$이다.
  \haskellpart{./src/ch11-ex06.hs}{80}{86}
  앞의 함수는 근접한 해답을 찾기 위해 다음의 $nearests$ 함수를 사용한다.
  \haskellpart{./src/ch11-ex06.hs}{75}{78}
  다음은 수정한 프로그램을 사용해 $1,4,10$에서 $8$을 만드는 계산식을
  찾아본 것이다. 결과에 따르면 어떠한 경우로도 $8$을 만들 수는
  없으며, 대신에 차이가 $1$인 $7$, $9$를 만드는 계산식이 존재하는 것을
  알 수 있다.
  \begin{lstlisting}
*Main> putSolutions (solutionsOrNearests [1,4,10] 8)
10 - 1 = 9
1 + (10 - 4) = 7
(1 + 10) - 4 = 7
10 - (1 ^ 4) = 9
10 - (4 - 1) = 7
  \end{lstlisting}
\item 계산식의 단순함을 나타내는 지표를 계산식에서 쓰인 수의 갯수로
  정하자. 다음은 이러한 기준대로 계산식의 단순함을 얻어내는 함수이다.
  \haskellpart{./src/ch11-ex06.hs}{123}{124}
  이제 앞의 함수가 내 준 수가 작은 순으로 계산식을 정렬하면 된다.
  다음은 $Data.List$ 라이브러리에서 제공하는 $sortBy$ 함수를 사용하여 주어진
  계산식 리스트를 정렬하는 함수이다.
  \haskellpart{./src/ch11-ex06.hs}{126}{127}
  이 함수를 사용하여 구해진 해답을 정렬하여 내주는 함수를 작성하면 된다.
  \haskellpart{./src/ch11-ex06.hs}{129}{135}
  다음은 $1,3,7,10,20$을 사용해 $30$을 만들어내는 계산식을 찾아본
  것이다. 계산식이 짧은 순으로 정렬되어 출력됨을 볼 수 있다.
  \begin{lstlisting}
*Main> putSolutions (sortedSolutionsOrNearests [1,3,7,10,20] 30)
10 + 20 = 30
3 * 10 = 30
3 * (20 - 10) = 30
3 + (7 + 20) = 30
(3 + 7) + 20 = 30
7 + (3 + 20) = 30
  ...
(10 / (1 ^ 7)) + 20 = 30
(10 ^ (1 ^ 7)) + 20 = 30
(10 * (1 + 20)) / 7 = 30
10 + (20 / (1 ^ 7)) = 30
  ...
  \end{lstlisting}
\end{enumerate}

%%% Local Variables: 
%%% mode: latex
%%% TeX-master: "master"
%%% End: 
      