\chapter{\Large{Types and classes}}

%%%%%%%%%%%%%%%%%%%%%%%%%%%%%%%%%%%%%%%%%%%%%%%%%%%%%%%%%%%%%%%%%%%%%%
\exercise{3.1}

\begin{itemize}
\item \texttt{['a', 'b', 'c'] :: [Char]}
\item \texttt{('a', 'b', 'c') :: (Char, Char, Char)}
\item \texttt{[(False, '0'), (True, '1')] :: [(Bool, Char)]}
\item \texttt{([False, True], ['0', '1']) :: ([Bool], [Char])}
\item \texttt{[tail, init, reverse] :: [[a] -> [a]]}
\end{itemize}

%%%%%%%%%%%%%%%%%%%%%%%%%%%%%%%%%%%%%%%%%%%%%%%%%%%%%%%%%%%%%%%%%%%%%%
\exercise{3.2}

\begin{itemize}
\item \texttt{second :: [a] -> a}
\item \texttt{swap :: (a, b) -> (b, a)}
\item \texttt{pair :: a -> b -> (a, b)}
\item \texttt{double :: Num a => a -> a}
\item \texttt{palindrome xs :: Eq a => [a] -> Bool}
\item \texttt{twice f x :: (a -> a) -> a -> a}
\end{itemize}

%%%%%%%%%%%%%%%%%%%%%%%%%%%%%%%%%%%%%%%%%%%%%%%%%%%%%%%%%%%%%%%%%%%%%%
\exercise{3.3}

(생략)

%%%%%%%%%%%%%%%%%%%%%%%%%%%%%%%%%%%%%%%%%%%%%%%%%%%%%%%%%%%%%%%%%%%%%%
\exercise{3.4}

어떤 두 함수가 같은 인자에 대해 같은 결과를 내놓을 때, 그 둘이 같은 함수라고
말한다고 하자. 이러한 규칙에 따라 두 함수가 같음을 비교하기 위해서는 우선
가능한 모든 인자에 대해 같은 결과를 내놓는지 비교해야 한다. 하지만 대부분의
경우 가능한 인자수는 너무나 많기 때문에, 이는 현실적으로 불가능하다. 게다가 어떤
훌륭한 프로그램이 존재하여 모든 인자에 대한 결과가
같음을 빠르게 비교할 수 있다고 하여도, 계산이 끝나지 않는 함수가 존재할 수
있으므로 여전히 이러한 함수에 대해서는 결과를 얻을 수 없다.


%%% Local Variables: 
%%% mode: latex
%%% TeX-master: "master"
%%% End: 
