\chapter{\Large{Lazy evaluation}}

%%%%%%%%%%%%%%%%%%%%%%%%%%%%%%%%%%%%%%%%%%%%%%%%%%%%%%%%%%%%%%%%%%%%%%
\exercise{12.1}
\begin{itemize}
\item $1 + (2 * 3)$
  \begin{itemize}
  \item $2 * 3$ : \KOEN{최내부}{innermost} \KOEN{줄임대상}{redex}
  \item $1 + (...)$ : \KOEN{최외부}{outermost} 줄임대상
  \end{itemize}
\item $(1 + 2) * (2 + 3)$
  \begin{itemize}
  \item $1 + 2$ : 최내부 줄임대상
  \item $2 + 3$ : 최내부 줄임대상
  \item $(...) * (...)$ : 최외부 줄임대상
  \end{itemize}
\item $fst~(1 + 2, 2 + 3)$
  \begin{itemize}
  \item $1 + 2$ : 최내부 줄임대상
  \item $2 + 3$ : 최내부 줄임대상
  \item $fst~(...)$ : 최외부 줄임대상
  \end{itemize}
\item $(\lambda x \ra 1 + x) (2 * 3)$
  \begin{itemize}
  \item $1 + x$ : 줄임대상이 아님 (람다 표현식 내부)
  \item $2 * 3$ : 최내부 줄임대상
  \item $(...) (...)$ (계산식 전체) : 최외부 줄임대상
  \end{itemize}
\end{itemize}


%%%%%%%%%%%%%%%%%%%%%%%%%%%%%%%%%%%%%%%%%%%%%%%%%%%%%%%%%%%%%%%%%%%%%%
\exercise{12.2}
주어진 계산식은 최내부 우선 방식으로는 다음과 같이 계산된다.
\begin{align*}
    & fst~(1+2,2+3) \\
  = & \qquad \{~ \text{최내부 줄임대상 중 왼쪽 것, $1+2$ 부터 계산} ~\} \\
    & fst~(3,2+3) \\
  = & \qquad \{~ \text{$2+3$ 계산} ~\} \\
    & fst~(3,5) \\
  = & \qquad \{~ \text{$fst$ 계산} ~\} \\
    & 3
\end{align*}

계산 과정을 살펴보면, 첫번째 계산이 끝나면 바로 전체 답을 구할 수 있음에도
답과 관련없는 $2+3$을 계산하는 것을 볼 수 있다. 이와 달리 최외부 우선 방식을
사용하면 다음과 같이 두 번만에 계산을 완료할 수 있어 더 효율적이다.
\begin{align*}
    & fst~(1+2,2+3) \\
  = & \qquad \{~ \text{최외부 줄임대상, $fst$ 계산} ~\} \\
    & 1+2 \\
  = & \qquad \{~ \text{$1+2$ 계산} ~\} \\
    & 3
\end{align*}


%%%%%%%%%%%%%%%%%%%%%%%%%%%%%%%%%%%%%%%%%%%%%%%%%%%%%%%%%%%%%%%%%%%%%%
\exercise{12.3}
\begin{align*}
    & mult~3~4 \\
  = & \qquad \{~ \text{$mult$에 $3$을 적용} ~\} \\
    & (\lambda y \ra 3 * y)~4 \\
  = & \qquad \{~ \text{람다 표현식에 $4$을 적용} ~\} \\
    & 3*4 \\
  = & \qquad \{~ \text{$3*4$ 계산} ~\} \\
    & 12
\end{align*}


%%%%%%%%%%%%%%%%%%%%%%%%%%%%%%%%%%%%%%%%%%%%%%%%%%%%%%%%%%%%%%%%%%%%%%
\exercise{12.4}
피보나치 수열에서 $i$번째 수를 $f_i$라고 하자. 그러면 피보나치 수열은 다음과
같이 재귀적으로 구할 수 있다.
\begin{equation*}
\begin{array}{cccccccr}
  &   &   & f_1 & f_2 & f_3 & \cdots & \quad\text{(재귀적으로 고려한 피보나치 수열)} \\
+ &   &   & f_2 & f_3 & f_4 & \cdots & \text{(위의 수열에 $tail$을 적용)}\\
\hline
  & 0 & 1 & f_3 & f_4 & f_5 & \cdots & \text{(결과 피보나치 수열)}
\end{array}
\end{equation*}

피보나치 수열을 $fib$라 할 때, 두번째 줄의 식은 $tail~fib$로 구할 수 있으며,
최종 결과에서 $f_2$ 이후는 $tail~fib$와 $fib$를 $zip$으로 묶은 후 각 원소를
더하면 된다. 이와 같은 방식을 사용, 다음과 같이 간단하게 프로그램을 작성할 수
있다.
\haskell{./src/ch12-ex04.hs}

%%%%%%%%%%%%%%%%%%%%%%%%%%%%%%%%%%%%%%%%%%%%%%%%%%%%%%%%%%%%%%%%%%%%%%
\exercise{12.5}

$n$번째 피보나치 수를 구하려면 피보나치 수열 리스트에서 앞의 $n-1$개 원소를
버린 후 남은 리스트의 맨 앞을 꺼내면 된다.
\haskell[4]{./src/ch12-ex05.hs}

피보나치 수열에서 $1000$보다 큰 수 중 맨 처음 등장하는 수를 구하려면, 다음과
같이 $dropWhile$을 사용하여 $1000$보다 작은 수를 버린 후 남은 리스트에서 맨
앞을 꺼내오면 된다.
\begin{lstlisting}
*Main> head (dropWhile (< 1000) fibs)
1597
\end{lstlisting}

%%%%%%%%%%%%%%%%%%%%%%%%%%%%%%%%%%%%%%%%%%%%%%%%%%%%%%%%%%%%%%%%%%%%%%
\exercise{12.6}
\haskell{./src/ch12-ex06.hs}



%%% Local Variables: 
%%% mode: latex
%%% TeX-master: "master"
%%% End: 