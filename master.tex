\documentclass[final,a4wide,11pt]{report}
\usepackage[hangul]{kotex}
\usepackage{a4wide}
\usepackage{pdfsync}
\usepackage{ifpdf}
\ifpdf
  \usepackage[unicode,pdftex,colorlinks]{hyperref}
  \input glyphtounicode\pdfgentounicode=1
\else
  \usepackage[unicode,divpdfm,colorlinks]{hyperref}
\fi

\usepackage[T1]{fontenc}
\usepackage[scaled]{beramono}

\usepackage{amsmath, amssymb}
\usepackage{graphics}
\usepackage{listings}

\usepackage{graphicx}
\graphicspath{{./res/}}

\newcommand{\tit}[0]{\textit}
\newcommand{\tsf}[0]{\textsf}
\newcommand{\ttt}[0]{\texttt}
\newcommand{\tsc}[0]{\textsc}
\newcommand{\tbf}[0]{\textbf}
\newcommand{\TT}[0]{\texttt}

\newcommand{\mbb}[0]{\mathbb}
\newcommand{\mcal}[0]{\mathcal}

%%%
%%% Arrows
%%%
\newcommand{\ra}[0]{\rightarrow}
\newcommand{\Ra}[0]{\Rightarrow}
\newcommand{\rhu}[0]{\rightharpoonup}
\newcommand{\la}[0]{\leftarrow}
\newcommand{\La}[0]{\Leftarrow}
\newcommand{\Lra}[0]{\Leftrightarrow}
\newcommand{\IFF}[0]{\Longleftrightarrow}

%%%
%%% Quotes
%%%
\newcommand{\ql}{\textquoteleft}
\newcommand{\qr}{\textquoteright}
\newcommand{\Ql}{\textquotedblleft}
\newcommand{\Qr}{\textquotedblright}

\newcommand{\PAIR}[2]{\langle #1,#2 \rangle}

\newcommand{\FLR}[1]{\lfloor #1 \rfloor}
\newcommand{\CEIL}[1]{\lceil #1 \rceil}

\newcommand{\bs}{\backslash}

\newcommand{\set}[1]{\{#1\}}
\newcommand{\CAS}[1]{[\![#1]\!]}

\newcommand{\URL}[1]{\href{#1}{\texttt{#1}}}
\newcommand{\EML}[1]{\href{mailto:#1}{\texttt{#1}}}

\newcommand{\KOEN}[2]{$\text{#1}^{\text{#2}}$}

\newcommand{\TODO}{\noindent\textcolor{red}{\framebox[2\width][c]{\texttt{TODO}}}}

\newcommand{\QED}{\hfill \ensuremath{\Box}}

%%% Beamer %%%%%%%%%%%%%%%%%%%%%%%%%%%%%%%%%%%%%%%%%%%%%%%%%%%%%%%%%%%%%%%%%%%

\newenvironment{blt}{\begin{itemize}\item}{\end{itemize}}
\newcommand{\ftitle}[1]{\frametitle{\textsf{#1}}}


%%% drawing %%%%%%%%%%%%%%%%%%%%%%%%%%%%%%%%%%%%%%%%%%%%%%%%%%%%%%%%%%%%%%%%%%

\newcommand\HR[2][2]{\rule[#1pt]{#2}{0.4pt}}


%%% fancyvrb %%%%%%%%%%%%%%%%%%%%%%%%%%%%%%%%%%%%%%%%%%%%%%%%%%%%%%%%%%%%%%%%%

\makeatletter
\@ifpackageloaded{fancyvrb}
{
\DefineVerbatimEnvironment{code}%
  {Verbatim}{baselinestretch=1.0,fontsize=\small,numbers=left}
}
{}
\makeatother

\makeatletter
\@ifpackageloaded{listings}
{
\lstset{
  basicstyle=\small\ttfamily,
  keywordstyle=\ttfamily\bfseries,
  commentstyle=\color[gray]{0.4},
  numbers=left,
  numberstyle=\tiny,
  numbersep=10pt,
}
\lstdefinelanguage{Scheme}[]{Lisp}
  {keywords={define,if,cond,else,and,or,not}
  }
\newcommand{\scheme}[2][1]{
  \lstinputlisting[language=Scheme,firstline=#1]{#2}
}
\newcommand{\schemepart}[3]{
\lstinputlisting[language=Scheme,firstline=#2,lastline=#3]{#1}
}
\newcommand{\ocaml}[2][1]{
  \lstinputlisting[language={[Objective]Caml},firstline=#1]{#2}
}
\newcommand{\ocamlpart}[3]{
\lstinputlisting[language={[Objective]Caml},firstline=#2,lastline=#3]{#1}
}
}
{}
\makeatother

%%% graphics %%%%%%%%%%%%%%%%%%%%%%%%%%%%%%%%%%%%%%%%%%%%%%%%%%%%%%%%%%%%%%%%%

\newcommand{\imgfile}[3]{
\begin{figure}[#2]
  \centering
  \includegraphics[scale=#3]{#1}
\end{figure}
}

\title{
\textsc{Programming in Haskell}\vspace{15pt}\\
\includegraphics[scale=0.9]{cover}\\
\textsf{연습문제 해답}\\
}
\author{
\href{mailto:scjung@pav.hanyang.ac.kr}{정승철} \\
\href{http://pav.hanyang.ac.kr}{한양대학교 프로그램 분석검증 연구실}
}
\date{
마지막 고침\\
\today
}

\newcommand{\GHC}[0]{\textsf{GHC}}
\newcommand{\HASKELL}[0]{\textsf{Haskell}}
\newcommand{\GTK}[0]{\textsf{GTK}}
\newcommand{\OCAML}[0]{\textsf{Objective-Caml}}

\newcommand{\exercise}[1]{%
\vspace{1em}\section*{Exercise #1}%
\addcontentsline{toc}{section}{Exercise #1}%
}

\newcounter{myenumi}
\newenvironment{MyEnum}
  {\begin{list}{\alph{myenumi}.}%
      {\setlength\leftmargin{0pt}%
       \usecounter{myenumi}}}%
  {\end{list}}

\begin{document}

\maketitle
\tableofcontents
\chapter{\Large{Introduction}}

%%%%%%%%%%%%%%%%%%%%%%%%%%%%%%%%%%%%%%%%%%%%%%%%%%%%%%%%%%%%%%%%%%%%%%
\exercise{1.1}

\begin{lstlisting}[language=Haskell,escapeinside=~~]
  double (double 2)
=    { ~바깥쪽 double을 적용~ }
  double 2 + double 2
=    { ~두번째 double을 적용~ }
  double 2 + (2 + 2)
=    { ~첫번째 double을 적용~ }
  (2 + 2) + (2 + 2)
=    { ~첫번째 +를 적용~ }
  4 + (2 + 2)
=    { ~두번째 +를 적용~ }
  4 + 4
=    { ~+를 적용~ }
  8
\end{lstlisting}

%%%%%%%%%%%%%%%%%%%%%%%%%%%%%%%%%%%%%%%%%%%%%%%%%%%%%%%%%%%%%%%%%%%%%%
\exercise{1.2}

다음은 `\texttt{sum [x]}'가 계산되는 과정을 나타낸 것이다.

\begin{lstlisting}[language=Haskell,escapeinside=~~]
  sum [x]
=    { ~sum의 두번째 경우에 적용되며 이 때 x는 x, xs는 [].~ }
  x + sum []
=    { ~sum의 첫번째 경우에 적용.~ }
  x + 0
=    { ~+를 적용.~ }
  x
\end{lstlisting}

%%%%%%%%%%%%%%%%%%%%%%%%%%%%%%%%%%%%%%%%%%%%%%%%%%%%%%%%%%%%%%%%%%%%%%
\exercise{1.3}

다음은 구현된 \texttt{product} 함수이다. 첫번째 줄은 \GHC에서 기본으로 제공되는
\texttt{Prelude} 모듈 내 \texttt{product} 함수와의 이름 충돌을 피하기 위한
구문이다.

\haskell{./src/ch01-ex03.hs}

다음은 `\texttt{product [2,3,4]}'가 계산되는 과정을 나타낸 것이다. 정답인
`\texttt{24}'를 결과로 내준다.

\begin{lstlisting}[language=Haskell]
  product [2,3,4]
= 2 * product [3,4]
= 2 * 3 * product [4]
= 2 * 3 * 4 * product []
= 2 * 3 * 4 * 1
= 2 * 3 * 4
= 2 * 12
= 24
\end{lstlisting}

%%%%%%%%%%%%%%%%%%%%%%%%%%%%%%%%%%%%%%%%%%%%%%%%%%%%%%%%%%%%%%%%%%%%%%
\exercise{1.4}

교재의 구현에서 \KOEN{기준}{pivot}, 기준보다 작은 것, 기준보다 큰 것을
나열하여 합치는 순서를 거꾸로 하면 최종결과로 뒤집혀진 리스트가
나온다. 다시말해 두 번째 줄을 `\texttt{qsort larger ++ [x] ++ qsort
  smaller}'로 바꾸면 된다. 다음은 이러한 구현이 실제로 적용되는 모습을
나타낸 것이다.

\begin{lstlisting}[language=Haskell]
  qsort [3,5,1,4,2]
= qsort [5,4] ++ [3] ++ qsort [1,2]
= (qsort [] ++ [5] ++ qsort [4]) ++ [3] ++ (qsort [2] ++ [1] ++ qsort [])
= ([] ++ [5] ++ [4]) ++ [3] ++ ([2] ++ [1] ++ [])
= [5,4] ++ [3] ++ [2,1]
= [5,4,3,2,1]
\end{lstlisting}

%%%%%%%%%%%%%%%%%%%%%%%%%%%%%%%%%%%%%%%%%%%%%%%%%%%%%%%%%%%%%%%%%%%%%%
\exercise{1.5}

코드에서 `\texttt{<=}'를 `\texttt{<}'로 바꾸면 주어진 리스트에 중복된 값이 여럿
있을때, 기준과 같은 값이 결과 리스트에서 빠지는 문제가 발생한다. 예를 들면
다음과 같다.

\begin{lstlisting}[language=Haskell,escapeinside=~~]
  qsort [2,2,3,1,1]
=     { ~smaller에서 두번째 2가 제외됨~ }
  qsort [1,1] ++ [2] ++ qsort [3]
=     { ~smaller에서 두번째 1이 제외됨~ }
= (qsort [] ++ [1] ++ qsort []) ++ [2] ++ [3]
= [1] ++ [2] ++ [3]
= [1,2,3]
\end{lstlisting}


%%% Local Variables: 
%%% mode: latex
%%% TeX-master: "master"
%%% End: 

\chapter{\Large{First Steps}}

%%%%%%%%%%%%%%%%%%%%%%%%%%%%%%%%%%%%%%%%%%%%%%%%%%%%%%%%%%%%%%%%%%%%%%
\exercise{2.1}

\begin{lstlisting}[language=Haskell,escapeinside=~~]
(2 ^ 3) * 4
(2 * 3) + (4 * 5)
2 + (3 * (4 ^ 5))
\end{lstlisting}

%%%%%%%%%%%%%%%%%%%%%%%%%%%%%%%%%%%%%%%%%%%%%%%%%%%%%%%%%%%%%%%%%%%%%%
\exercise{2.2}

(생략)

%%%%%%%%%%%%%%%%%%%%%%%%%%%%%%%%%%%%%%%%%%%%%%%%%%%%%%%%%%%%%%%%%%%%%%
\exercise{2.3}

주어진 스크립트에는 다음 세가지 문제점이 있다.
\begin{itemize}
\item 함수 이름이 소문자로 시작되지 않는다.
\item 함수를 인자 사이에서 쓰려면 따옴표(\texttt{'})가 아닌
  역따옴표(\texttt{`})로 묶어야 한다.
\item \texttt{where} 다음에 오는 문장의 들여쓰기가 일치하지 않는다.
\end{itemize}

수정된 스크립트는 다음과 같다.

\haskell{./src/ch02-ex03.hs}

%%%%%%%%%%%%%%%%%%%%%%%%%%%%%%%%%%%%%%%%%%%%%%%%%%%%%%%%%%%%%%%%%%%%%%
\exercise{2.4}

\haskell{./src/ch02-ex04.hs}


%%%%%%%%%%%%%%%%%%%%%%%%%%%%%%%%%%%%%%%%%%%%%%%%%%%%%%%%%%%%%%%%%%%%%%
\exercise{2.5}

\haskell{./src/ch02-ex05.hs}


%%% Local Variables: 
%%% mode: latex
%%% TeX-master: "master"
%%% End: 

\chapter{\Large{Types and classes}}

%%%%%%%%%%%%%%%%%%%%%%%%%%%%%%%%%%%%%%%%%%%%%%%%%%%%%%%%%%%%%%%%%%%%%%
\exercise{3.1}

\begin{itemize}
\item \texttt{['a', 'b', 'c'] :: [Char]}
\item \texttt{('a', 'b', 'c') :: (Char, Char, Char)}
\item \texttt{[(False, '0'), (True, '1')] :: [(Bool, Char)]}
\item \texttt{([False, True], ['0', '1']) :: ([Bool], [Char])}
\item \texttt{[tail, init, reverse] :: [[a] -> [a]]}
\end{itemize}

%%%%%%%%%%%%%%%%%%%%%%%%%%%%%%%%%%%%%%%%%%%%%%%%%%%%%%%%%%%%%%%%%%%%%%
\exercise{3.2}

\begin{itemize}
\item \texttt{second :: [a] -> a}
\item \texttt{swap :: (a, b) -> (b, a)}
\item \texttt{pair :: a -> b -> (a, b)}
\item \texttt{double :: Num a => a -> a}
\item \texttt{palindrome xs :: Eq a => [a] -> Bool}
\item \texttt{twice f x :: (a -> a) -> a -> a}
\end{itemize}

%%%%%%%%%%%%%%%%%%%%%%%%%%%%%%%%%%%%%%%%%%%%%%%%%%%%%%%%%%%%%%%%%%%%%%
\exercise{3.3}

(생략)

%%%%%%%%%%%%%%%%%%%%%%%%%%%%%%%%%%%%%%%%%%%%%%%%%%%%%%%%%%%%%%%%%%%%%%
\exercise{3.4}

어떤 두 함수가 같은 인자에 대해 같은 결과를 내놓을 때, 그 둘이 같은 함수라고
말한다고 하자. 이러한 규칙에 따라 두 함수가 같음을 비교하기 위해서는 우선
가능한 모든 인자에 대해 같은 결과를 내놓는지 비교해야 한다. 하지만 대부분의
경우 가능한 인자수는 너무나 많기 때문에, 이는 현실적으로 불가능하다. 게다가 어떤
훌륭한 프로그램이 존재하여 모든 인자에 대한 결과가
같음을 빠르게 비교할 수 있다고 하여도, 계산이 끝나지 않는 함수가 존재할 수
있으므로 여전히 이러한 함수에 대해서는 결과를 얻을 수 없다.


%%% Local Variables: 
%%% mode: latex
%%% TeX-master: "master"
%%% End: 

\chapter{\Large{Defining functions}}

%%%%%%%%%%%%%%%%%%%%%%%%%%%%%%%%%%%%%%%%%%%%%%%%%%%%%%%%%%%%%%%%%%%%%%
\exercise{4.1}

\haskell{./src/ch04-ex01.hs}


%%%%%%%%%%%%%%%%%%%%%%%%%%%%%%%%%%%%%%%%%%%%%%%%%%%%%%%%%%%%%%%%%%%%%%
\exercise{4.2}

\haskell{./src/ch04-ex02.hs}


%%%%%%%%%%%%%%%%%%%%%%%%%%%%%%%%%%%%%%%%%%%%%%%%%%%%%%%%%%%%%%%%%%%%%%
\exercise{4.3}

\haskell{./src/ch04-ex03.hs}


%%%%%%%%%%%%%%%%%%%%%%%%%%%%%%%%%%%%%%%%%%%%%%%%%%%%%%%%%%%%%%%%%%%%%%
\exercise{4.4}

\haskell{./src/ch04-ex04.hs}


%%%%%%%%%%%%%%%%%%%%%%%%%%%%%%%%%%%%%%%%%%%%%%%%%%%%%%%%%%%%%%%%%%%%%%
\exercise{4.5}

\haskell{./src/ch04-ex05.hs}


%%%%%%%%%%%%%%%%%%%%%%%%%%%%%%%%%%%%%%%%%%%%%%%%%%%%%%%%%%%%%%%%%%%%%%
\exercise{4.6}

주어진 함수를 람다표현식 형태로 기술하면 다음과 같다.
\begin{equation}\notag
  mult = \lambda x \ra (\lambda y \ra (\lambda z \ra x * y * z))
\end{equation}


%%% Local Variables: 
%%% mode: latex
%%% TeX-master: "master"
%%% End: 

\chapter{\Large{List comprehensions}}

%%%%%%%%%%%%%%%%%%%%%%%%%%%%%%%%%%%%%%%%%%%%%%%%%%%%%%%%%%%%%%%%%%%%%%
\exercise{5.1}


%%% Local Variables: 
%%% mode: latex
%%% TeX-master: "master"
%%% End: 

\chapter{\Large{Recursive functions}}

%%%%%%%%%%%%%%%%%%%%%%%%%%%%%%%%%%%%%%%%%%%%%%%%%%%%%%%%%%%%%%%%%%%%%%
\exercise{6.1}
\haskell{./src/ch06-ex01.hs}

%%%%%%%%%%%%%%%%%%%%%%%%%%%%%%%%%%%%%%%%%%%%%%%%%%%%%%%%%%%%%%%%%%%%%%
\exercise{6.2}
\begin{itemize}
\item $length~[1,2,3]$
  \begin{align*}
      & length~[1,2,3] \\
    = & \qquad \{~ \text{applying}~length ~\} \\
      & 1 + length~[2,3] \\
    = & \qquad \{~ \text{applying}~length ~\} \\
      & 1 + (1 + length~[3]) \\
    = & \qquad \{~ \text{applying}~length ~\} \\
      & 1 + (1 + (1 + length~[])) \\
    = & \qquad \{~ \text{applying}~length ~\} \\
      & 1 + (1 + (1 + 0)) \\
    = & \qquad \{~ \text{applying}~+ ~\} \\
      & 1 + (1 + 1) \\
    = & \qquad \{~ \text{applying}~+ ~\} \\
      & 1 + 2 \\
    = & \qquad \{~ \text{applying}~+ ~\} \\
      & 3
  \end{align*}
\item $drop~3~[1,2,3,4,5]$
  \begin{align*}
      & drop~3~[1,2,3,4,5] \\
    = & \qquad \{~ \text{applying}~drop ~\} \\
      & drop~2~[2,3,4,5] \\
    = & \qquad \{~ \text{applying}~drop ~\} \\
      & drop~1~[3,4,5] \\
    = & \qquad \{~ \text{applying}~drop ~\} \\
      & drop~0~[4,5] \\
    = & \qquad \{~ \text{applying}~drop ~\} \\
      & [4,5]
  \end{align*}
\item $init~[1,2,3]$
  \begin{align*}
      & init~[1,2,3] \\
    = & \qquad \{~ \text{applying}~init ~\} \\
      & 1:init~[2,3] \\
    = & \qquad \{~ \text{applying}~init ~\} \\
      & 1:(2:init~[3]) \\
    = & \qquad \{~ \text{applying}~init ~\} \\
      & 1:(2:[]) \\
    = & \qquad \{~ \text{applying}~: ~\} \\
      & 1:[2] \\
    = & \qquad \{~ \text{applying}~: ~\} \\
      & [1,2]
  \end{align*}
\end{itemize}

%%%%%%%%%%%%%%%%%%%%%%%%%%%%%%%%%%%%%%%%%%%%%%%%%%%%%%%%%%%%%%%%%%%%%%
\exercise{6.3}
다음은 구현된 코드이다. 대부분 \texttt{Prelude}에 정의된 함수와 이름 충돌이
발생하는데, 함수 이름 끝에 따옴표를 붙이거나 \texttt{Main} 모듈을 명시하여
문제를 해결하였다.

연산자 \texttt{!!}의 경우에는 주어진 수가 주어진 리스트의 인덱스 범위를
벗어나는 경우가 처리되지 않았다. 이러한 경우에는 적절한 예외를 발생시켜야
하는데, 아직 예외에 대해서는 다루지 않았으므로 여기서는 무시하고 지나가도록
한다. 사실 실제로 인덱스 범위를 벗어나는 경우를 시험해보면 (의미는 정확하지
않더라도) 처리되지 않은 패턴이 있다는 예외가 발생한다.
\haskell{./src/ch06-ex03.hs}

%%%%%%%%%%%%%%%%%%%%%%%%%%%%%%%%%%%%%%%%%%%%%%%%%%%%%%%%%%%%%%%%%%%%%%
\exercise{6.4}
\haskell{./src/ch06-ex04.hs}

%%%%%%%%%%%%%%%%%%%%%%%%%%%%%%%%%%%%%%%%%%%%%%%%%%%%%%%%%%%%%%%%%%%%%%
\exercise{6.5}
\haskell[7]{./src/ch06-ex05.hs}

%%%%%%%%%%%%%%%%%%%%%%%%%%%%%%%%%%%%%%%%%%%%%%%%%%%%%%%%%%%%%%%%%%%%%%
\exercise{6.6}
\begin{itemize}
\item $sum$
  \begin{enumerate}
  \item 타입 정의하기: 일단 단순히 정수의 리스트를 받아서 원소의 합을 내주는
    함수에서 시작하자. 함수의 타입은 다음과 같다.
    \[sum~::~[Int]~\rightarrow~Int\]
  \item 경우를 나열하기: 리스트를 받으므로, 가능한 경우는 빈 리스트인 경우와
    하나 이상의 원소가 있는 경우이다.
    \[\begin{array}{lcl}
      sum~[]       & = & \\
      sum~(x : xs) & = &
    \end{array}\]
  \item 간단한 경우를 정의하기: 우선 빈 리스트가 주어지는 경우를 고려하자. 이
    때의 합은 $0$으로 한다.
    \[\begin{array}{lcl}
      sum~[]       & = & 0 \\
      sum~(x : xs) & = &
    \end{array}\]
  \item 나머지 경우를 정의하기: 원소가 하나 이상인 경우에는, 재귀적으로
   정의한다. 뒷 리스트의 합을 함수 $sum$을 사용하여 재귀적으로 구하고, 이를 맨
   앞의 수와 더하면 된다.
    \[\begin{array}{lcl}
      sum~[]       & = & 0 \\
      sum~(x : xs) & = & x + sum~xs
    \end{array}\]
  \item 일반화 및 단순화 작업: 앞의 타입 정의에서는 주어진 리스트의 원소가
    $Int$ 타입이라고 하였으나, 사실 `더할' 수 있는 모든 것은 원소가 될 수
    있다. 그리고 어떤 타입의 값을 더하 결과 역시 이전 타입과 같다. 결과적으로
    다음과 같이 리스트 원소 및 결과 타입을 $Num$에 해당하는 타입으로 일반화
    시킬 수 있다.
    \[\begin{array}{lcl}
      sum          & :: & Num~a \Rightarrow a \rightarrow a \\
      sum~[]       &  = & 0 \\
      sum~(x : xs) &  = & x + sum~xs
    \end{array}\]
  \end{enumerate}

\item $take$
  \begin{enumerate}
  \item 타입 정의하기: 가져올 원소의 갯수에 해당하는 정수와 리스트를 받아서
    가져온 원소가 들어있는 리스트를 내준다. 주어진 리스트와 내놓을
    리스트의 타입은 같아야 한다. 이를 타입으로 정리하면 다음과 같다.
    \[take~::~Int~\rightarrow~[a]~\rightarrow~[a]\]
  \item 경우를 나열하기: 정수와 리스트를 받으므로, 가능한 모든 경우는 정수가
    $0$,$n+1$인 경우와 리스트가 빈 리스트, 하나 이상의 원소가 있는 경우
    각각을 짝지은 것이다. 정리하면 다음과 같다.
    \[\begin{array}{lcl}
      take~0~[]           & = & \\
      take~0~(x : xs)     & = & \\
      take~(n+1)~[]       & = & \\
      take~(n+1)~(x : xs) & = &
    \end{array}\]
  \item 간단한 경우를 정의하기: 주어진 정수가 0인 경우에는 아무것도 가져오지
    않으므로 결과는 빈 리스트이다. 또한 정수가 0보다 크고 리스트가 빈 리스트인
    경우에는 더 이상 가져올 것이 없으므로 이 경우 역시 빈 리스트이다.
    \[\begin{array}{lcl}
      take~0~[]           & = & [] \\
      take~0~(x : xs)     & = & [] \\
      take~(n+1)~[]       & = & [] \\
      take~(n+1)~(x : xs) & = &
    \end{array}\]
  \item 나머지 경우를 정의하기: 정수가 0보다 크고 원소가 하나 이상인 리스트가
    주어진 경우에는 함수 $take$를 재귀적으로 사용하여, 리스트 꼬리에서 원소를
    주어진 정수보다 하나 덜 가져온 후, 만들어진 리스트 맨 앞에 주어진 리스트의
    머리 원소를 붙이면 된다.
    \[\begin{array}{lcl}
      take~0~[]           & = & [] \\
      take~0~(x : xs)     & = & [] \\
      take~(n+1)~[]       & = & [] \\
      take~(n+1)~(x : xs) & = & x : take~n~xs
    \end{array}\]
  \item 일반화 및 단순화 작업: $Int$ 뿐만 아니라 어떠한 정수라도 리스트에서
    가져올 원소의 갯수를 나타내는 인자로 쓰일 수 있으므로, 이 인자의 타입을
    $Integral$에 해당하는 타입으로 더 일반화 시킬 수 있다.
    \[\begin{array}{lcl}
      take & :: & Integral~b~\Rightarrow~b~\rightarrow~[a]~\rightarrow~[a]\\
      take~0~[]           & = & [] \\
      take~0~(x : xs)     & = & [] \\
      take~(n+1)~[]       & = & [] \\
      take~(n+1)~(x : xs) & = & x : take~n~xs
    \end{array}\]
    또한 첫 인자가 0일 경우에는 어떠한 리스트가 오더라도 결과가 같으므로
    아래와 같이 경우의 수를 줄일 수 있다.
    \[\begin{array}{lcl}
      take & :: & Integral~b~\Rightarrow~b~\rightarrow~[a]~\rightarrow~[a]\\
      take~0~\_           & = & [] \\
      take~(n+1)~[]       & = & [] \\
      take~(n+1)~(x : xs) & = & x : take~n~xs
    \end{array}\]
  \end{enumerate}

\item $last$
  \begin{enumerate}
  \item 타입 정의하기: 리스트를 받고 리스트의 마지막 원소를 내준다. 따라서
    타입은 다음과 같다.
    \[last~::~[a]~\rightarrow~a\]
  \item 경우를 나열하기: 리스트를 받으므로, 가능한 경우는 빈 리스트인 경우와
    하나 이상의 원소가 있는 경우이다. 그런데 빈 리스트는 가능한 입력이 아니므로
    코드에서 제외한다.
    \[\begin{array}{lcl}
      last~(x : xs) & = &
    \end{array}\]
  \item 간단한 경우를 정의하기: 제시된 경우는 다시 꼬리 리스트가 빈 리스트인
    경우와 아닌 경우로 나눌 수 있다. 전자의 경우에는 맨 앞 원소 $x$가 마지막
    원소이다.
    \[\begin{array}{lclcl}
      last~(x : xs) &|& null~xs   & = & x\\
                    &|& otherwise & = & \\
    \end{array}\]
  \item 나머지 경우를 정의하기: 꼬리 리스트가 빈 리스트가 아닌 경우에는 꼬리
    리스트에서 마지막 원소를 재귀적으로 찾으면 된다.
    \[\begin{array}{lclcl}
      last~(x : xs) &|&null~xs   & = & x\\
                    &|&otherwise & = & last~xs\\
    \end{array}\]
  \item 일반화 및 단순화 작업: 앞에서 쓰인 가드를 패턴 매칭으로 바꾸어
    정리하면 다음과 같다.
    \[\begin{array}{lcl}
      last & :: & [a]~\rightarrow~a \\
      last~[x]      & = & x \\
      last~(\_ : xs) & = & last~xs \\
    \end{array}\]
  \end{enumerate}
\end{itemize}


%%% Local Variables: 
%%% mode: latex
%%% TeX-master: "master"
%%% End: 
\chapter{\Large{Higher-order functions}}

%%%%%%%%%%%%%%%%%%%%%%%%%%%%%%%%%%%%%%%%%%%%%%%%%%%%%%%%%%%%%%%%%%%%%%
\exercise{7.1}
주어진 리스트 조건표현식은 다음과 같은 표현식과 동일하다.
\begin{lstlisting}[language=Haskell]
(map f) . (filter p)
\end{lstlisting}

%%%%%%%%%%%%%%%%%%%%%%%%%%%%%%%%%%%%%%%%%%%%%%%%%%%%%%%%%%%%%%%%%%%%%%
\exercise{7.2}
\haskell{./src/ch07-ex02.hs}

%%%%%%%%%%%%%%%%%%%%%%%%%%%%%%%%%%%%%%%%%%%%%%%%%%%%%%%%%%%%%%%%%%%%%%
\exercise{7.3}
\haskell{./src/ch07-ex03.hs}

%%%%%%%%%%%%%%%%%%%%%%%%%%%%%%%%%%%%%%%%%%%%%%%%%%%%%%%%%%%%%%%%%%%%%%
\exercise{7.4}
\haskell{./src/ch07-ex04.hs}

%%%%%%%%%%%%%%%%%%%%%%%%%%%%%%%%%%%%%%%%%%%%%%%%%%%%%%%%%%%%%%%%%%%%%%
\exercise{7.5}
리스트 내 모든 원소의 타입은 동일해야 하는데, 주어진 정의에서 $compose$의 인자로
쓰인 리스트 내 함수 중 함수 $sum$은 나머지 함수와 타입이 다르다. 나머지 함수는
정수 리스트를 내놓는 반면에, $sum$은 정수를 내놓는다.

%%%%%%%%%%%%%%%%%%%%%%%%%%%%%%%%%%%%%%%%%%%%%%%%%%%%%%%%%%%%%%%%%%%%%%
\exercise{7.6}
\haskell{./src/ch07-ex06.hs}

%%%%%%%%%%%%%%%%%%%%%%%%%%%%%%%%%%%%%%%%%%%%%%%%%%%%%%%%%%%%%%%%%%%%%%
\exercise{7.7}
\haskell[10]{./src/ch07-ex07.hs}

%%%%%%%%%%%%%%%%%%%%%%%%%%%%%%%%%%%%%%%%%%%%%%%%%%%%%%%%%%%%%%%%%%%%%%
\exercise{7.8}
우선 다음과 같이 비트 리스트를 받아서 이에 해당하는 패리티 비트를 계산하는 함수
`\texttt{parity}'를 작성한다.
\haskellpart{./src/ch07-ex08-09.hs}{18}{19}

패리티 비트는 각 8비트 단위 리스트 맨 앞에 붙이도록 하자. 다음은 이 작업을
수행하는 함수 `\texttt{addParity}'이다.
\haskellpart{./src/ch07-ex08-09.hs}{21}{22}

다음과 같이 `\texttt{encode}' 함수를 수정하여 8비트 단위로 쪼갠 후 맨
앞에 패리티 비트가 붙도록 한다.
\haskellpart{./src/ch07-ex08-09.hs}{25}{25}

다음은 문자열 ``\texttt{Bye}''을 변경된 함수로 인코딩한 것이다. 패리티 비트를
쉽게 구별해내기 위해 의도적으로 9비트 단위별로 행을 나누었다.
\begin{lstlisting}
*Main> encode "Bye"
[0,0,1,0,0,0,0,1,0,
 1,1,0,0,1,1,1,1,0,
 0,1,0,1,0,0,1,1,0]
\end{lstlisting}

이제 패리티 비트가 추가된 비트 데이터를 다시 문자열로 복원하는 프로그램을
작성해 보자. 전과 달리 이제는 9비트가 한 문자를 나타내므로, 일단 비트를 쪼개는
함수 `\texttt{chop8}'을 수정해야 한다.
\haskellpart{./src/ch07-ex08-09.hs}{27}{28}

앞의 함수로 쪼개진 9비트 단위 리스트에서 패리티 비트가 올바른지를 검사하고
데이터 영역만을 떼어내어 주는 함수 `\texttt{dataOfChunk}'를 작성한다.
\haskellpart{./src/ch07-ex08-09.hs}{30}{32}

마지막으로 다음과 같이 `\texttt{decode}' 함수를 수정하면 된다.
\haskellpart{./src/ch07-ex08-09.hs}{35}{35}

%%%%%%%%%%%%%%%%%%%%%%%%%%%%%%%%%%%%%%%%%%%%%%%%%%%%%%%%%%%%%%%%%%%%%%
\exercise{7.9}

패리티 비트 기능이 동작하는지 확인하기 위해 다음과 같이 의도적으로 잘못된 비트
데이터를 전송하는 채널과, 그 채널을 사용하여 문자열을 전송하는 함수를 추가하자.
\haskellpart{./src/ch07-ex08-09.hs}{43}{47}

다음은 작성한 함수로 문자열을 전송해 본 것이다. 데이터가 잘못 전송되었음을
나타내는 오류가 발생하는 것을 볼 수 있다.
\begin{lstlisting}[language=Haskell]
*Main> erroneousTransmit "Haskell"
*** Exception: incorrect parity bit
\end{lstlisting}



%%% Local Variables: 
%%% mode: latex
%%% TeX-master: "master"
%%% End: 
\chapter{\Large{Functional parsers}}

%%%%%%%%%%%%%%%%%%%%%%%%%%%%%%%%%%%%%%%%%%%%%%%%%%%%%%%%%%%%%%%%%%%%%%
\exercise{8.1}
\haskellpart{./src/ch08-ex01.hs}{55}{59}

%%%%%%%%%%%%%%%%%%%%%%%%%%%%%%%%%%%%%%%%%%%%%%%%%%%%%%%%%%%%%%%%%%%%%%
\exercise{8.2}
\haskellpart{./src/ch08-ex02.hs}{43}{49}

%%%%%%%%%%%%%%%%%%%%%%%%%%%%%%%%%%%%%%%%%%%%%%%%%%%%%%%%%%%%%%%%%%%%%%
\exercise{8.3}
그림 \ref{fig:ex83}은 제시된 표현식에서 가능한 두 가지의 파스 트리를 도식화한
것이다. 이렇게 여러 파스 트리가 가능한 이유는 $+$ 연산자가 어느 방향으로
결합되는지를 정하지 않았기 때문이다.
\begin{figure}[t]
  \centering
  \includegraphics[width=0.8\textwidth]{ch08-ex03}
  \caption{$2+3+4$에 대한 두 가지 파스 트리}
  \label{fig:ex83}
\end{figure}

%%%%%%%%%%%%%%%%%%%%%%%%%%%%%%%%%%%%%%%%%%%%%%%%%%%%%%%%%%%%%%%%%%%%%%
\exercise{8.4}
그림 \ref{fig:ex84}은 제시된 표현식 각각에 해당하는 파스 트리를 도식화한 것이다.
\begin{figure}[t]
  \centering
  \includegraphics[width=1\textwidth]{ch08-ex04}
  \caption{$2+3+4$, $2 * 3 * 4$, $(2+3)+4$에 대한 파스 트리}
  \label{fig:ex84}
\end{figure}

%%%%%%%%%%%%%%%%%%%%%%%%%%%%%%%%%%%%%%%%%%%%%%%%%%%%%%%%%%%%%%%%%%%%%%
\exercise{8.5}
마지막 단순화 작업 바로 전의 파서로 수 하나를 파싱한다고 가정해보자. 이 때는
다음과 같은 순서로 파싱이 진행된다. 우선 $expr$의 첫번째 경우, $term+expr$로
파싱을 시도한다. 처음에는 이 문법의 첫번째 항 $term$이 파싱되고, 이어서 두번째
항 $+$를 파싱하게 되는데, 이 때는 이미 입력이 모두 처리된 상태이므로 파싱은
실패하게 된다. 연이어 $expr$의 두번째 경우 $term$을 시도하며, 이 때는 파싱이
성공하게 된다. 이러한 순서를 자세히 살펴보면, 수 하나 전체가 첫번째 경우와
두번째 경우의 $term$에 의해 중복되어 파싱됨을 알 수 있다.

이에 반해, 단순화 작업을 거친 문법에 대해 수 하나를 파싱하는 경우에는 중복
작업이 발생하지 않는다. 이 때는 $expr$ 문법의 처음 항 $term$에 의해
주어진 수 전체가 파싱된 후, 다음 항이 무시되면서 전체 파싱이 끝나기
때문이다. 즉, 중복된 항을 하나로 합침으로서 파싱이 실패하였을 때 다시 반복하는
경우가 줄어들게 되고, 이에 따라 파서의 성능이 향상되게 된다.

%%%%%%%%%%%%%%%%%%%%%%%%%%%%%%%%%%%%%%%%%%%%%%%%%%%%%%%%%%%%%%%%%%%%%%
\exercise{8.6}
\haskellpart{./src/ch08-ex06.hs}{91}{109}

%%%%%%%%%%%%%%%%%%%%%%%%%%%%%%%%%%%%%%%%%%%%%%%%%%%%%%%%%%%%%%%%%%%%%%
\exercise{8.7}
다음과 같이 문법을 확장한다.

\[\begin{array}{lcl}
  expr     & ::= & term~(+~expr~|~-~expr~|~\epsilon) \\
  term     & ::= & exponent~(*~expr~|~/~expr~|~\epsilon) \\
  exponent & ::= & (exponent~\uparrow~|~\epsilon)~factor \\
  factor   & ::= & (expr) ~|~ nat \\
  nat      & ::= & 0 ~|~ 1 ~|~ 2 ~|~ \cdots
\end{array}\]

이에 따라 앞에서 작성한 프로그램을 다음과 같이 수정한다.
\haskellpart{./src/ch08-ex07.hs}{101}{116}

%%%%%%%%%%%%%%%%%%%%%%%%%%%%%%%%%%%%%%%%%%%%%%%%%%%%%%%%%%%%%%%%%%%%%%
\exercise{8.8}
\renewcommand{\theenumi}{\alph{enumi}}
\renewcommand{\labelenumi}{(\theenumi)}
\begin{enumerate}
\item 문법은 다음과 같이 정의할 수 있다.
  \[\begin{array}{lcl}
    expr     & ::= & expr~-~nat ~|~ nat \\
    nat      & ::= & 0 ~|~ 1 ~|~ 2 ~|~ \cdots
  \end{array}\]
\item 앞의 문법을 그대로 코드로 나타내면 다음과 같다.
  \begin{lstlisting}[language=Haskell]
expr :: Parser Int
expr  = do e <- expr
           symbol "-"
           n <- natural
           return (e - n)
          +++ natural
  \end{lstlisting}
\item 이 프로그램은 $expr$을 파싱할 시 맨 처음에 $expr$ 파서를 사용하기
  때문에, 결국에는 무한한 재귀호출이 일어나게 된다.
\item $expr$ 문법을 다시 생각해보면, 결국에는 `$nat~-$' 형태가 없거나 한 번 이상
  나온 뒤 $nat$으로 끝나는 것과 같다. 따라서 `$nat~-$' 형태에 대한 파서를 따로
  만들고, 그 파서를 $many$로 감싼 후 파싱하면 간단히 구현할 수 있다. 이렇게
  되면 파싱 결과는 정수의 리스트가 되므로, 최종 계산 결과는 $foldl$을 사용하여
  얻어낼 수 있다. (리스트에는 계산식의 수가 거꾸로 들어있게 됨을 유의)
  \begin{lstlisting}[language=Haskell]
expr :: Parser Int
expr  = do es <- many expr1
           n <- natural
           case es of
             [] -> return n
             h : t -> return (foldl (-) h (n : t))

expr1 :: Parser Int
expr1  = do e <- natural
            symbol "-"
            return e
  \end{lstlisting}
\end{enumerate}


%%% Local Variables: 
%%% mode: latex
%%% TeX-master: "master"
%%% End: 

\chapter{\Large{Interactive programs}}

%%%%%%%%%%%%%%%%%%%%%%%%%%%%%%%%%%%%%%%%%%%%%%%%%%%%%%%%%%%%%%%%%%%%%%
\exercise{9.1}

기존의 $getLine$ 함수를 사용하여 입력할 때, 삭제 키를 누르면 (ANSI
터미널에서는) `\texttt{\^{}?}'와 같은 문자열이 입력되어 버린다. 이를 제대로
처리하기 위해서는 입력이 `\texttt{\textbackslash DEL}' 경우를 구분하여 다음과
같이 처리해줘야 한다.

\begin{itemize}
\item $getLine$ 함수에서 쓰인 $getChar$ 함수는 문자가 입력되자마자 그 문자가
  화면에 출력되는데, 삭제 키에 대한 문자는 화면에 출력되어선 안되므로
  $readLine$ 함수에서 쓰기엔 적합하지 않다. 대신에
  $getCh$ 함수를 사용하고, 입력이 `\texttt{\textbackslash DEL}'이 아닌
  경우에만 입력을 화면에 출력하도록 수정한다.
\item 만일 입력이 `\texttt{\textbackslash DEL}'이라면, 바로 전까지 입력된
  문자열에서 가장 마지막의 입력을 버리고서 계속 입력을 받아야 한다. 즉,
  작성하고자하는 함수는 이제까지 입력된 문자열이 무엇인지 알아야 한다. 따라서
  보조함수 $readLine1 :: String \rightarrow IO String$을 작성하고, $readLine$
  함수는 이 보조함수를 사용하도록 한다.
\item 입력이 `\texttt{\textbackslash DEL}' 일 때는 (1) 우선 뒤로 한 칸 움직이고,
  (2) 빈 공백 하나를 출력하여 문자를 지운 후, (3) 다시 뒤로 한 칸 움직인 다음,
  (4) 이제까지 입력된 문자열에서 꼬리만을 가지고 $readLine1$을 재귀 호출하면 된다.
\end{itemize}

다음은 구현된 코드이다.

\haskell[10]{./src/ch09-ex01.hs}

%%%%%%%%%%%%%%%%%%%%%%%%%%%%%%%%%%%%%%%%%%%%%%%%%%%%%%%%%%%%%%%%%%%%%%
\exercise{9.2}

파싱이 실패한 지점 바로 위에 `\texttt{v}'를 출력하여 오류 위치를 쉽게 파악할
수 있게 해보자. 우선 다음과 같이 오류 위치를 표시하는 함수와 표시된 위치를
없애는 함수를 작성한다.

\haskellpart{./src/ch09-ex02.hs}{164}{170}

그리고 나서 $eval$ 함수에서 계산이 실패한 경우에 파싱되지 못한 문자열을 통해
오류 위치를 계산한 후 표시하도록 수정한다.

\haskellpart{./src/ch09-ex02.hs}{200}{205}

마지막으로 표시된 오류 위치가 계속 남아있으면 보기에 좋지 않으므로, 특정 키가
눌리면 이전에 표시한 오류 위치가 사라지도록 $calc$ 함수를 수정한다.

\haskellpart{./src/ch09-ex02.hs}{172}{180}

다음은 구현된 계산기에서 잘못된 계산식인 ``\texttt{3+4//6}''을 입력하고 리턴을
누른 후의 모습 중 일부이다. 적절한 오류 위치가 표시됨을 볼 수 있다.

\begin{lstlisting}
+----v----------+
| 3+4//6        |
\end{lstlisting}

%%%%%%%%%%%%%%%%%%%%%%%%%%%%%%%%%%%%%%%%%%%%%%%%%%%%%%%%%%%%%%%%%%%%%%
\exercise{9.3}

문제에서 제시된 방식대로 구현하기 위해서는 게임판을 그릴 때 바로 이전의
게임판 또한 알고 있어야 한다. 따라서 새로운 보조 함수 $life1~::~Board
\rightarrow Board \rightarrow IO~()$을 작성하고 $life$는 이 함수를 사용하도록
수정한다. 그리고 실제 게임판을 그리는 함수인 $showcells$ 역시 이전 게임판 또한
인자로 받도록 수정한다.

다음은 수정된 $showcells$ 함수이다. 이전 게임판에서 없다가 이번 게임판에서
새로 생긴 위치($bc$)와 이전엔 있다가 이번에 사라진 위치($dc$) 각각을 찾고,
전자에 대해서는 `\texttt{o}'를, 후자에 대해서는 빈 칸을 출력한다.

\haskellpart{./src/ch09-ex03.hs}{29}{32}

앞의 함수를 사용하여 재귀적으로 게임을 수행하는 보조 함수 $life1$을
작성한다. 화면을 지우는 $cls$ 함수를 호출하지 않음에 유의한다.

\haskellpart{./src/ch09-ex03.hs}{71}{75}

마지막으로 함수 $life$를 다음과 같이 수정한다.

\haskellpart{./src/ch09-ex03.hs}{67}{69}

%%%%%%%%%%%%%%%%%%%%%%%%%%%%%%%%%%%%%%%%%%%%%%%%%%%%%%%%%%%%%%%%%%%%%%
\exercise{9.4}

%%%%%%%%%%%%%%%%%%%%%%%%%%%%%%%%%%%%%%%%%%%%%%%%%%%%%%%%%%%%%%%%%%%%%%
\exercise{9.5}

%%%%%%%%%%%%%%%%%%%%%%%%%%%%%%%%%%%%%%%%%%%%%%%%%%%%%%%%%%%%%%%%%%%%%%
\exercise{9.6}


%%% Local Variables: 
%%% mode: latex
%%% TeX-master: "master"
%%% End: 

\chapter{\Large{Declaring types and classes}}

%%%%%%%%%%%%%%%%%%%%%%%%%%%%%%%%%%%%%%%%%%%%%%%%%%%%%%%%%%%%%%%%%%%%%%
\exercise{10.1}
\haskell[20]{./src/ch10-ex01.hs}

%%%%%%%%%%%%%%%%%%%%%%%%%%%%%%%%%%%%%%%%%%%%%%%%%%%%%%%%%%%%%%%%%%%%%%
\exercise{10.2}
\haskell[3]{./src/ch10-ex02.hs}

이전 구현은 주어진 트리가 노드인 경우, 가능한 경우마다 따로 값 비교를
수행한다. 따라서 최악의 경우에는 주어진 수가 노드의 수보다 크다는 사실을 알기
위해 같음, 작음, 큼을 세 번 비교하게 된다. 하지만 여기서 새로 작성한
코드는 대소여부를 판별한 결과가 세 경우 각각을 나타내므로 $compare$를
여러번 사용할 필요가 없어, 더 효율적이다.

%%%%%%%%%%%%%%%%%%%%%%%%%%%%%%%%%%%%%%%%%%%%%%%%%%%%%%%%%%%%%%%%%%%%%%
\exercise{10.3}
\haskell[5]{./src/ch10-ex03.hs}

%%%%%%%%%%%%%%%%%%%%%%%%%%%%%%%%%%%%%%%%%%%%%%%%%%%%%%%%%%%%%%%%%%%%%%
\exercise{10.4}
\haskell[5]{./src/ch10-ex04.hs}

%%%%%%%%%%%%%%%%%%%%%%%%%%%%%%%%%%%%%%%%%%%%%%%%%%%%%%%%%%%%%%%%%%%%%%
\exercise{10.5}
우선 다음과 같이 새 요소를 데이터 선언에 추가한다.
\haskellpart{./src/ch10-ex05.hs}{6}{12}

이에 따라 $eval$ 함수에 계산 코드를 추가한다.
\haskellpart{./src/ch10-ex05.hs}{22}{23}

마지막으로 $vars$ 함수에 변수를 찾는 코드를 추가한다.
\haskellpart{./src/ch10-ex05.hs}{31}{32}

%%%%%%%%%%%%%%%%%%%%%%%%%%%%%%%%%%%%%%%%%%%%%%%%%%%%%%%%%%%%%%%%%%%%%%
\exercise{10.6}

먼저 작성할 프로그램이 받는 입력의 문법을 정의해야 한다. 우선 각
수학 기호에 해당하는 실제 입력 문자열 토큰을 정의해보자.

\begin{center}
  \begin{tabular}{cc}
    수학 기호          & 입력 문자열 \\
    \hline
    $T$               & \texttt{T} \\
    $F$               & \texttt{F} \\
    $\neg$            & \texttt{!} \\
    $\wedge$          & \texttt{/\textbackslash} \\
    $\vee$            & \texttt{\textbackslash/} \\
    $\Rightarrow$     & \texttt{=>} \\
    $\Leftrightarrow$ & \texttt{<=>} \\
  \end{tabular}
\end{center}

다음은 정의에 따라 작성된 요소 각각을 파싱하는 함수이다. 8장에서 사용된 코드를
재사용하고 있음에 유의하라.
\haskellpart{./src/ch10-ex06.hs}{126}{145}

이제 이러한 요소로 어떻게 명제가 구성되는지를 나타내는 문법을 정의하자. 문법을
정의할 때는 모호한 경우가 발생하지 않도록 우선순위, 결합방향을 주의깊게
고려해야 한다. 결합방향은 모든 경우에 대해 우결합으로 하고\footnote{즉,
  `\texttt{a => b => c}'는 `\texttt{a => (b => c)}'와 같다.}, 우선순위는 제일
높은 순으로 $\neg \rightarrow \wedge, \vee \rightarrow \Rightarrow,
\Leftrightarrow$의 순서를 따른다. 정의된 문법은 다음과 같다.

\[\begin{array}{lcl}
   prop & ::= & term ~\texttt{=>}~ prop ~|~ term ~\texttt{<=>}~ prop ~|~ term \\
   term & ::= & not ~\texttt{/\textbackslash} ~term ~|~ not ~\texttt{\textbackslash/} ~term ~|~ not \\
    not & ::= & \texttt{!} ~not ~|~ paren \\
  paren & ::= & \texttt{(} ~prop~ \texttt{)} ~|~ const ~|~ var \\
  const & ::= & \texttt{T} ~|~ \texttt{F} \\
    var & ::= & \texttt{a} ~|~ \texttt{b} ~|~ \cdots
\end{array}\]

변수명은 소문자로 이루어진다.\footnote{대문자로도 이루어지게 하면, 참과 거짓을
  뜻하는 \texttt{T}와 \texttt{F}가 변수로도 해석될 수 있다. 여기서는 이런
  모호한 경우를 피하기 위해 변수는 반드시 소문자로만 쓰이도록 하였다.} 이제
정의한 문법에 따라 파서 코드를 작성하면 다음과 같다.

\haskellpart{./src/ch10-ex06.hs}{147}{189}

다음은 사용자의 입력을 받아 토톨로지 여부를 출력해주는 사용자 인터페이스
코드이다.

\haskell[191]{./src/ch10-ex06.hs}

다음은 교재에서 제시된 명제가 항진식인지를 작성된 프로그램을 통해 확인해 본
모습이다.

\begin{lstlisting}
*Main> taut
tautology? a /\ !a
no
tautology? (a /\ b) => a
yes
tautology? a => (a /\ b)
no
tautology? (a /\ (a => b)) => b
yes
\end{lstlisting}

%%%%%%%%%%%%%%%%%%%%%%%%%%%%%%%%%%%%%%%%%%%%%%%%%%%%%%%%%%%%%%%%%%%%%%
\exercise{10.7}
\haskell{./src/ch10-ex07.hs}

%%%%%%%%%%%%%%%%%%%%%%%%%%%%%%%%%%%%%%%%%%%%%%%%%%%%%%%%%%%%%%%%%%%%%%
\exercise{10.8}
\haskell{./src/ch10-ex08.hs}


%%% Local Variables: 
%%% mode: latex
%%% TeX-master: "master"
%%% End: 
      
\chapter{\Large{The countdown problem}}

%%%%%%%%%%%%%%%%%%%%%%%%%%%%%%%%%%%%%%%%%%%%%%%%%%%%%%%%%%%%%%%%%%%%%%
\exercise{11.1}
\haskellpart{./src/ch11-ex01.hs}{17}{18}

%%%%%%%%%%%%%%%%%%%%%%%%%%%%%%%%%%%%%%%%%%%%%%%%%%%%%%%%%%%%%%%%%%%%%%
\exercise{11.2}
\haskell{./src/ch11-ex02.hs}

%%%%%%%%%%%%%%%%%%%%%%%%%%%%%%%%%%%%%%%%%%%%%%%%%%%%%%%%%%%%%%%%%%%%%%
\exercise{11.3}
만일 $split$을 다음과 같이 빈 리스트가 쌍에 포함되도록 수정한다면,
\begin{lstlisting}
split         :: [a] -> [([a], [a])]
split []       = []
split [x]      = [([x], [])]
split (x : xs) = ([], (x : xs)) : ([x], xs) : [(x : ls, rs) | (ls, rs) <- split xs]
\end{lstlisting}

\noindent
결과에는 쌍의 왼쪽 혹은 오른쪽 원소가 나누려는 리스트 전체인 쌍이 포함되게
된다. 따라서 $results$ 함수의 리스트 조건식 내 생성자에서 쓰이는 재귀호출 중
하나는 원래의 리스트에 대해 다시 호출되어, 이에 따라 재귀호출이 무한히 일어나게
된다. 따라서 $solutions$ 계산은 끝나지 않는다.

%%%%%%%%%%%%%%%%%%%%%%%%%%%%%%%%%%%%%%%%%%%%%%%%%%%%%%%%%%%%%%%%%%%%%%
\exercise{11.4}
구성할 수 있는 모든 계산식의 수는 다음과 같이 구할 수 있다.
\begin{lstlisting}
*Main> length [e | ns' <- choices [1,3,7,10,25,50], e <- exprs ns']
33665406
\end{lstlisting}

이 중 실제로 계산할 수 있는 계산식의 수는 다음과 같이 구할 수 있다.
\begin{lstlisting}
*Main> length [r | ns' <- choices [1,3,7,10,25,50], e <- exprs ns', r <- eval e]
4672540
\end{lstlisting}

%%%%%%%%%%%%%%%%%%%%%%%%%%%%%%%%%%%%%%%%%%%%%%%%%%%%%%%%%%%%%%%%%%%%%%
\exercise{11.5}
다음과 같이 $valid$ 함수를 수정한다. 이 때 정수만이 취급되어야 하므로 $Div$의
경우에서 나누어 떨어지는지 여부를 검사하는 것은 지우지 말아야 한다. 또한
$0$으로 나누는 경우가 발생하지 않도록 추가적으로 검사를 수행해야 한다.

\begin{lstlisting}[language=Haskell]
valid        :: Op -> Int -> Int -> Bool
valid Add _ _ = True
valid Sub _ _ = True
valid Mul _ _ = True
valid Div x y = y /= 0 && x `mod` y == 0
\end{lstlisting}

다음과 같이 실제로 계산할 수 있는 계산식의 수를 구할 수 있다.
\begin{lstlisting}
*Main> length [r | ns <- choices [1,3,7,10,25,50], e <- exprs ns, r <- eval e]
10839369
\end{lstlisting}


%%%%%%%%%%%%%%%%%%%%%%%%%%%%%%%%%%%%%%%%%%%%%%%%%%%%%%%%%%%%%%%%%%%%%%
\exercise{11.6}

\begin{enumerate}
\item 제곱을 사용한 계산식을 허용하기 위해, 우선 연산자 정의를 다음과
  같이 수정한다.
  \haskellpart{./src/ch11-ex06.hs}{4}{4}
  이에 따라 관련된 함수인 $valid$, $apply$에 제곱에 관한 경우를 추가해준다.
  \haskellpart{./src/ch11-ex06.hs}{7}{19}
  마지막으로 해답을 구할 때 쓰이는 $ops$ 리스트에 제곱 연산자를 추가한다.
  \haskellpart{./src/ch11-ex06.hs}{67}{68}
  다음은 수정한 프로그램을 사용해 $1,4,6,20$에서 $100$을 만들 수 있는
  계산식을 알아본 것이다. 여기서 쓰인 $putSolutions$ 함수는 찾아낸 $Expr$
  리스트를 보기 쉽게 출력해주는 $Expr \rightarrow IO~()$ 타입의 함수이다.
  \begin{lstlisting}
*Main> putSolutions (solutions [1,4,6,20] 100)
(6 - 1) * 20 = 100
(1 + 4) * 20 = 100
4 * ((6 - 1) + 20) = 100
4 * (6 + (20 - 1)) = 100
4 * ((6 + 20) - 1) = 100
((1 ^ 6) + 4) * 20 = 100
(6 - (1 ^ 4)) * 20 = 100
  \end{lstlisting}
\item 다음은 정확한 해답을 찾지 못할 때 근접한 해답을 대신 내주는 함수
  $solutionsOrNearests$이다.
  \haskellpart{./src/ch11-ex06.hs}{80}{86}
  앞의 함수는 근접한 해답을 찾기 위해 다음의 $nearests$ 함수를 사용한다.
  \haskellpart{./src/ch11-ex06.hs}{75}{78}
  다음은 수정한 프로그램을 사용해 $1,4,10$에서 $8$을 만드는 계산식을
  찾아본 것이다. 결과에 따르면 어떠한 경우로도 $8$을 만들 수는
  없으며, 대신에 차이가 $1$인 $7$, $9$를 만드는 계산식이 존재하는 것을
  알 수 있다.
  \begin{lstlisting}
*Main> putSolutions (solutionsOrNearests [1,4,10] 8)
10 - 1 = 9
1 + (10 - 4) = 7
(1 + 10) - 4 = 7
10 - (1 ^ 4) = 9
10 - (4 - 1) = 7
  \end{lstlisting}
\item 계산식의 단순함을 나타내는 지표를 계산식에서 쓰인 수의 갯수로
  정하자. 다음은 이러한 기준대로 계산식의 단순함을 얻어내는 함수이다.
  \haskellpart{./src/ch11-ex06.hs}{123}{124}
  이제 앞의 함수가 내 준 수가 작은 순으로 계산식을 정렬하면 된다.
  다음은 $Data.List$ 라이브러리에서 제공하는 $sortBy$ 함수를 사용하여 주어진
  계산식 리스트를 정렬하는 함수이다.
  \haskellpart{./src/ch11-ex06.hs}{126}{127}
  이 함수를 사용하여 구해진 해답을 정렬하여 내주는 함수를 작성하면 된다.
  \haskellpart{./src/ch11-ex06.hs}{129}{135}
  다음은 $1,3,7,10,20$을 사용해 $30$을 만들어내는 계산식을 찾아본
  것이다. 계산식이 짧은 순으로 정렬되어 출력됨을 볼 수 있다.
  \begin{lstlisting}
*Main> putSolutions (sortedSolutionsOrNearests [1,3,7,10,20] 30)
10 + 20 = 30
3 * 10 = 30
3 * (20 - 10) = 30
3 + (7 + 20) = 30
(3 + 7) + 20 = 30
7 + (3 + 20) = 30
  ...
(10 / (1 ^ 7)) + 20 = 30
(10 ^ (1 ^ 7)) + 20 = 30
(10 * (1 + 20)) / 7 = 30
10 + (20 / (1 ^ 7)) = 30
  ...
  \end{lstlisting}
\end{enumerate}

%%% Local Variables: 
%%% mode: latex
%%% TeX-master: "master"
%%% End: 
      

\end{document}

%%% Local Variables: 
%%% mode: latex
%%% TeX-master: t
%%% End: 
