\chapter{\Large{Introduction}}

%%%%%%%%%%%%%%%%%%%%%%%%%%%%%%%%%%%%%%%%%%%%%%%%%%%%%%%%%%%%%%%%%%%%%%
\exercise{1.1}

\begin{lstlisting}[language=Haskell,escapeinside=~~]
  double (double 2)
=    { ~바깥쪽 double을 적용~ }
  double 2 + double 2
=    { ~두번째 double을 적용~ }
  double 2 + (2 + 2)
=    { ~첫번째 double을 적용~ }
  (2 + 2) + (2 + 2)
=    { ~첫번째 +를 적용~ }
  4 + (2 + 2)
=    { ~두번째 +를 적용~ }
  4 + 4
=    { ~+를 적용~ }
  8
\end{lstlisting}

%%%%%%%%%%%%%%%%%%%%%%%%%%%%%%%%%%%%%%%%%%%%%%%%%%%%%%%%%%%%%%%%%%%%%%
\exercise{1.2}

다음은 `\texttt{sum [x]}'가 계산되는 과정을 나타낸 것이다.

\begin{lstlisting}[language=Haskell,escapeinside=~~]
  sum [x]
=    { ~sum의 두번째 경우에 적용되며 이 때 x는 x, xs는 [].~ }
  x + sum []
=    { ~sum의 첫번째 경우에 적용.~ }
  x + 0
=    { ~+를 적용.~ }
  x
\end{lstlisting}

%%%%%%%%%%%%%%%%%%%%%%%%%%%%%%%%%%%%%%%%%%%%%%%%%%%%%%%%%%%%%%%%%%%%%%
\exercise{1.3}

다음은 구현된 \texttt{product} 함수이다. 첫번째 줄은 \GHC에서 기본으로 제공되는
\texttt{Prelude} 모듈 내 \texttt{product} 함수와의 이름 충돌을 피하기 위한
구문이다.

\haskell{./src/ch01-ex03.hs}

다음은 `\texttt{product [2,3,4]}'가 계산되는 과정을 나타낸 것이다. 정답인
`\texttt{24}'를 결과로 내준다.

\begin{lstlisting}[language=Haskell]
  product [2,3,4]
= 2 * product [3,4]
= 2 * 3 * product [4]
= 2 * 3 * 4 * product []
= 2 * 3 * 4 * 1
= 2 * 3 * 4
= 2 * 12
= 24
\end{lstlisting}

%%%%%%%%%%%%%%%%%%%%%%%%%%%%%%%%%%%%%%%%%%%%%%%%%%%%%%%%%%%%%%%%%%%%%%
\exercise{1.4}

교재의 구현에서 \KOEN{기준}{pivot}, 기준보다 작은 것, 기준보다 큰 것을
나열하여 합치는 순서를 거꾸로 하면 최종결과로 뒤집혀진 리스트가
나온다. 다시말해 두 번째 줄을 `\texttt{qsort larger ++ [x] ++ qsort
  smaller}'로 바꾸면 된다. 다음은 이러한 구현이 실제로 적용되는 모습을
나타낸 것이다.

\begin{lstlisting}[language=Haskell]
  qsort [3,5,1,4,2]
= qsort [5,4] ++ [3] ++ qsort [1,2]
= (qsort [] ++ [5] ++ qsort [4]) ++ [3] ++ (qsort [2] ++ [1] ++ qsort [])
= ([] ++ [5] ++ [4]) ++ [3] ++ ([2] ++ [1] ++ [])
= [5,4] ++ [3] ++ [2,1]
= [5,4,3,2,1]
\end{lstlisting}

%%%%%%%%%%%%%%%%%%%%%%%%%%%%%%%%%%%%%%%%%%%%%%%%%%%%%%%%%%%%%%%%%%%%%%
\exercise{1.5}

코드에서 `\texttt{<=}'를 `\texttt{<}'로 바꾸면 주어진 리스트에 중복된 값이 여럿
있을때, 기준과 같은 값이 결과 리스트에서 빠지는 문제가 발생한다. 예를 들면
다음과 같다.

\begin{lstlisting}[language=Haskell,escapeinside=~~]
  qsort [2,2,3,1,1]
=     { ~smaller에서 두번째 2가 제외됨~ }
  qsort [1,1] ++ [2] ++ qsort [3]
=     { ~smaller에서 두번째 1이 제외됨~ }
= (qsort [] ++ [1] ++ qsort []) ++ [2] ++ [3]
= [1] ++ [2] ++ [3]
= [1,2,3]
\end{lstlisting}


%%% Local Variables: 
%%% mode: latex
%%% TeX-master: "master"
%%% End: 
