\chapter{\Large{Reasoning About Programs}}

%%%%%%%%%%%%%%%%%%%%%%%%%%%%%%%%%%%%%%%%%%%%%%%%%%%%%%%%%%%%%%%%%%%%%%
\exercise{13.1}

예를 들어 함수 $null$을 다음과 같이 패턴이 \KOEN{겹치도록}{overlapping} 정의할
수 있다.

\begin{equation*}
\begin{array}{lcl}
  null ~[] & = & True \\
  null ~\_ & = & False \\
\end{array}
\end{equation*}

정의된 두번째 패턴의 인자는, 이전 패턴을 고려해야만 반드시 빈 리스트가 아님을
알 수 있다.


%%%%%%%%%%%%%%%%%%%%%%%%%%%%%%%%%%%%%%%%%%%%%%%%%%%%%%%%%%%%%%%%%%%%%%
\exercise{13.2}
% add n (Succ m) = Succ (add n m)
\begin{itemize}
% add Zero (Succ m) = Succ (add Zero m)
\item \KOEN{기초증명}{base case}
  \begin{align*}
      & add ~Zero ~(Succ ~m) \\
    = &     \quad \CMT{$add$ 적용} \\
      & Succ ~m \\
    = &     \quad \CMT{$m$에 $add$를 \KOEN{역적용}{unapplying}} \\
      & Succ ~(Add ~Zero ~m) \\
  \end{align*}

% add (Succ n) (Succ m) = Succ (add (Succ n) m)
\item \KOEN{귀납증명}{induction case}
  \begin{align*}
      & add ~(Succ ~n) ~(Succ ~m) \\
    = &     \quad \CMT{$add$ 적용} \\
      & Succ ~(add ~n ~(Succ ~m)) \\
    = &     \quad \CMT{\KOEN{귀납가정}{induction hypothesis} 적용} \\
      & Succ ~(Succ ~(add ~n ~m)) \\
    = &     \quad \CMT{$add$ 역적용} \\
      & Succ ~(add ~(Succ ~n) ~m)
  \end{align*}
\end{itemize}


%%%%%%%%%%%%%%%%%%%%%%%%%%%%%%%%%%%%%%%%%%%%%%%%%%%%%%%%%%%%%%%%%%%%%%
\exercise{13.3}
% add n m = add m n     with add n Zero = n
\begin{itemize}
\item 기초증명
  % add Zero m = add m Zero
  \begin{align*}
      & add ~Zero ~m \\
    = &     \quad \CMT{$add$ 적용} \\
      & m \\
    = &     \quad \CMT{주어진 성질 적용} \\
      & add ~m ~Zero
  \end{align*}

\item 귀납증명
  % add (Succ n) m = add m (Succ n)
  \begin{align*}
      & add ~(Succ ~n) ~m \\
    = &     \quad \CMT{$add$ 적용} \\
      & Succ ~(add ~n ~m) \\
    = &     \quad \CMT{귀납가정 적용} \\
      & Succ ~(add ~m ~n) \\
    = &     \quad \CMT{\textbf{13.2} 성질 적용} \\
      & add ~m ~(Succ ~n)
  \end{align*}
\end{itemize}


%%%%%%%%%%%%%%%%%%%%%%%%%%%%%%%%%%%%%%%%%%%%%%%%%%%%%%%%%%%%%%%%%%%%%%
\exercise{13.4}
% replicate 0 _     = []
% replicate (n+1) x = x : replicate n x
\begin{itemize}
\item 기초증명
  % all (== x) (replicate 0 x) = True
  \begin{align*}
      & all ~(== ~x) ~(replicate ~0 ~x) \\
    = &     \quad \CMT{$replicate$ 적용} \\
      & all ~(== ~x) ~[] \\
    = &     \quad \CMT{$all$ 적용} \\
      & True
  \end{align*}
\item 귀납증명
  % all (== x) (replicate (n+1) x) = True
  \begin{align*}
      & all ~(== ~x) ~(replicate ~(n+1) ~x) \\
    = &     \quad \CMT{$replicate$ 적용} \\
      & all ~(== ~x) ~(x ~: ~replicate ~n ~x) \\
    = &     \quad \CMT{$all$ 적용} \\
      & (x == x) \wedge (all ~(== ~x) ~(replicate ~n ~x)) \\
    = &     \quad \CMT{$==$ 적용} \\
      & True \wedge (all ~(== ~x) ~(replicate ~n ~x)) \\
    = &     \quad \CMT{귀납가정 적용} \\
      & True \wedge True \\
    = &     \quad \CMT{$\wedge$ 적용} \\
      & True
  \end{align*}
\end{itemize}


%%%%%%%%%%%%%%%%%%%%%%%%%%%%%%%%%%%%%%%%%%%%%%%%%%%%%%%%%%%%%%%%%%%%%%
\exercise{13.5}
\begin{itemize}
\item $xs \LAPP [] ~= ~xs$
  \begin{itemize}
  \item 기초증명
    % [] ++ [] = []
    \begin{align*}
        & [] ~\LAPP ~[] \\
      =~&     \quad \CMT{$\LAPP$ 적용} \\
        & []
    \end{align*}
  \item 귀납증명
    % (x : xs) ++ [] = (x : xs)
    \begin{align*}
        & (x : xs) ~\LAPP ~[] \\
      =~&     \quad \CMT{$\LAPP$ 적용} \\
        & x : (xs ~\LAPP ~[]) \\
      =~&     \quad \CMT{귀납가정 적용} \\
        & x : xs
    \end{align*}
  \end{itemize}
\item $xs ~\LAPP ~(ys ~\LAPP ~zs) ~= ~(xs ~\LAPP ~ys) ~\LAPP ~zs$
  \begin{itemize}
  \item 기초증명
    % [] ++ (ys ++ zs) = ([] ++ ys) ++ zs
    \begin{align*}
        & [] ~\LAPP ~(ys ~\LAPP ~zs) \\
      =~&     \quad \CMT{$\LAPP$ 적용} \\
        & ys ~\LAPP ~zs \\
      =~&     \quad \CMT{$\LAPP$ 첫번째 경우를 역적용} \\
        & ([] ~\LAPP ~ys) ~\LAPP ~zs
    \end{align*}
  \item 귀납증명
    % (x : xs) ++ (ys ++ zs) = ((x : xs) ++ ys) ++ zs
    \begin{align*}
        & (x ~: ~xs) ~\LAPP ~(ys ~\LAPP ~zs) \\
      =~&     \quad \CMT{$\LAPP$ 적용} \\
        & x ~: ~(xs ~\LAPP ~(ys ~\LAPP ~zs)) \\
      =~&     \quad \CMT{귀납가정 적용} \\
        & x ~: ~((xs ~\LAPP ~ys) ~\LAPP ~zs) \\
      =~&     \quad \CMT{$\LAPP$ 두번째 경우를 역적용} \\
        & (x ~: (xs ~\LAPP ~ys)) ~\LAPP ~zs \\
      =~&     \quad \CMT{$\LAPP$ 두번째 경우를 역적용} \\
        & ((x ~: ~xs) ~\LAPP ~ys) ~\LAPP ~zs \\
    \end{align*}
  \end{itemize}
\end{itemize}


%%%%%%%%%%%%%%%%%%%%%%%%%%%%%%%%%%%%%%%%%%%%%%%%%%%%%%%%%%%%%%%%%%%%%%
\exercise{13.6}
문제에서 주어진 \KOEN{보조결과}{auxilary result}는 증명하고자 하는 정의에만
국한되어 쓰이는 성질이다. 반면에 이전에 사용한 보조결과는 좀 더 일반적인
성질로서, 다른 성질을 증명하는데에도 유용하게 쓰일 수 있다.


%%%%%%%%%%%%%%%%%%%%%%%%%%%%%%%%%%%%%%%%%%%%%%%%%%%%%%%%%%%%%%%%%%%%%%
\exercise{13.7}
% map f []     = []
% map f (x:xs) = f x : map f xs
% (f . g) x    = f (g x)

% map f (map g xs) = map (f . g) xs
\begin{itemize}
\item 기초증명
  % map f (map g []) = map (f . g) []
  \begin{align*}
      & map ~f ~(map ~g ~[]) \\
    =~&     \quad \CMT{$map$ 적용} \\
      & map ~f ~[] \\
    =~&     \quad \CMT{$map$ 적용} \\
      & [] \\
    =~&     \quad \CMT{$map$ 첫번째 경우를 역적용} \\
      & map ~(f \circ g) ~[]
  \end{align*}

\item 귀납증명
  % map f (map g (x : xs)) = map (f . g) (x : xs)
  \begin{align*}
      & map ~f ~(map ~g ~(x ~: xs)) \\
    =~&     \quad \CMT{$map$ 적용} \\
      & map ~f ~(g ~x ~: ~map ~g ~xs) \\
    =~&     \quad \CMT{$map$ 적용} \\
      & f ~(g ~x) ~: ~map ~f ~(map ~g ~xs) \\
    =~&     \quad \CMT{귀납가정 적용} \\
      & f ~(g ~x) ~: ~map ~(f \circ g) ~xs \\
    =~&     \quad \CMT{$\circ$ 역적용} \\
      & (f \circ g) ~x ~: ~map ~(f \circ g) ~xs \\
    =~&     \quad \CMT{$map$ 역적용} \\
      & map ~(f \circ g) ~(x ~: ~xs)
  \end{align*}
\end{itemize}


%%%%%%%%%%%%%%%%%%%%%%%%%%%%%%%%%%%%%%%%%%%%%%%%%%%%%%%%%%%%%%%%%%%%%%
\exercise{13.8}
\begin{itemize}
\item $n$에 대한 기초증명
  % take 0 xs ++ drop 0 xs = xs
  \begin{align*}
      & take ~0 ~xs ~\LAPP ~drop ~0 ~xs \\
    =~&     \quad \CMT{$take$ 적용} \\
      & [] ~\LAPP ~drop ~0 ~xs \\
    =~&     \quad \CMT{$drop$ 적용} \\
      & [] ~\LAPP ~[] \\
    =~&     \quad \CMT{$\LAPP$ 적용} \\
      & []
  \end{align*}
\item $xs$에 대한 기초증명
  % take n [] ++ drop n [] = []
  \begin{align*}
      & take ~n ~[] ~\LAPP ~drop ~n ~[] \\
    =~&     \quad \CMT{$take$ 적용} \\
      & [] ~\LAPP ~drop ~n ~[] \\
    =~&     \quad \CMT{$drop$ 적용} \\
      & [] ~\LAPP ~[] \\
    =~&     \quad \CMT{$\LAPP$ 적용} \\
      & []
  \end{align*}
\item 귀납증명
  % take (n + 1) (x : xs) ++ drop (n + 1) (x : xs) = x : xs
  \begin{align*}
      & take ~(n+1) ~(x:xs) ~\LAPP ~drop ~(n+1) ~(x:xs) \\
    =~&     \quad \CMT{$take$ 적용} \\
      & (x:take ~n ~xs) ~\LAPP ~drop ~(n+1) ~(x:xs) \\
    =~&     \quad \CMT{$drop$ 적용} \\
      & (x:take ~n ~xs) ~\LAPP ~drop ~n ~xs \\
    =~&     \quad \CMT{$\LAPP$ 적용} \\
      & x : (take ~n ~xs ~\LAPP ~drop ~n ~xs) \\
    =~&     \quad \CMT{귀납가정 적용} \\
      & x : xs
  \end{align*}
\end{itemize}


%%%%%%%%%%%%%%%%%%%%%%%%%%%%%%%%%%%%%%%%%%%%%%%%%%%%%%%%%%%%%%%%%%%%%%
\exercise{13.9}
우선 다음과 같이 주어진 \KOEN{트리}{Tree}의 \KOEN{잎새}{leaf} 수를 세는 함수를
작성한다.
\begin{equation*}
  \begin{array}{lcl}
    leaves ~(Leaf ~\_)   &=& 1 \\
    leaves ~(Node ~l ~r) &=& leaves ~l + leaves ~r \\
  \end{array}
\end{equation*}

그리고 주어진 트리의 \KOEN{노드}{node} 수를 세는 함수도 작성한다.
\begin{equation*}
  \begin{array}{lcl}
    nodes ~(Leaf ~\_)   &=& 0 \\
    nodes ~(Node ~l ~r) &=& 1 + nodes ~l + nodes ~r \\
  \end{array}
\end{equation*}

이제 이렇게 정의된 함수를 바탕으로 `항상 트리 내 잎새는 노드보다 1만큼 더
많다'를 증명할 것이다. 증명할 명제를 수식으로 나타내면 다음과 같다.
\begin{equation*}
  nodes ~t + 1 = leaves ~t
\end{equation*}

증명은 다음과 같다.
\begin{itemize}
\item 기초증명
  % nodes (Leaf n) + 1 = leaves (Leaf n)
  \begin{align*}
      & nodes ~(Leaf ~n) + 1\\
    =~&     \quad \CMT{$nodes$ 적용} \\
      & 0 + 1 \\
    =~&     \quad \CMT{$leaves$ 첫번째 경우를 역적용} \\
      & 0 + leaves ~(Leaf n) \\
    =~&     \quad \CMT{$+$ 적용} \\
      & leaves ~(Leaf n)
  \end{align*}
\item 귀납증명
  % nodes (Node l r) + 1 = leaves (Node l r)
  \begin{align*}
      & nodes ~(Node ~l ~r) + 1\\
    =~&     \quad \CMT{$nodes$ 적용} \\
      & (1 + nodes ~l + nodes ~r) + 1 \\
    =~&     \quad \CMT{식 정렬} \\
      & (nodes ~l + 1) + (nodes ~r + 1) \\
    =~&     \quad \CMT{귀납가정 적용} \\
      & leaves ~l + leaves ~r \\
    =~&     \quad \CMT{$leaves$ 두번째 경우를 역적용} \\
      & leaves ~(Node ~l ~r) \\
  \end{align*}
\end{itemize}


%%%%%%%%%%%%%%%%%%%%%%%%%%%%%%%%%%%%%%%%%%%%%%%%%%%%%%%%%%%%%%%%%%%%%%
\exercise{13.10}
% comp' e c = comp e ++ c
\begin{itemize}
\item 기초증명
  % comp' (Val n) c = comp (Val n) ++ c
  \begin{align*}
      & comp' ~(Val ~n) ~c\\
    =~&     \quad \CMT{$comp'$ \KOEN{명세}{specification}} \\
      & comp ~(Val ~n) ~\LAPP~ c \\
    =~&     \quad \CMT{$comp$ 적용} \\
      & [PUSH ~n] ~\LAPP~ c \\
    =~&     \quad \CMT{$\LAPP$ 적용} \\
      & PUSH ~n : c
  \end{align*}
\item 귀납증명
  % comp' (Add x y) c = comp (Add x y) ++ c
  \begin{align*}
      & comp' ~(Add ~x ~y) ~c\\
    =~&     \quad \CMT{$comp'$ 명세} \\
      & comp ~(Add ~x ~y) ~\LAPP~ c \\
    =~&     \quad \CMT{$comp$ 적용} \\
      & (comp ~x ~\LAPP ~comp ~y ~\LAPP~ [ADD]) ~\LAPP~ c \\
    =~&     \quad \CMT{$\LAPP$ 적용} \\
      & comp ~x ~\LAPP ~comp ~y ~\LAPP~ (ADD : c) \\
    =~&     \quad \CMT{귀납가정 적용} \\
      & comp ~x ~\LAPP ~comp' ~y ~(ADD : c) \\
    =~&     \quad \CMT{귀납가정 적용} \\
      & comp' ~x ~(comp' ~y ~(ADD : c)) \\
  \end{align*}
\end{itemize}

즉, $comp'$는 다음과 같이 정의할 수 있다.
\begin{equation*}
  \begin{array}{lcl}
    comp' ~(Val ~n) ~c    &=& PUSH ~n : c \\
    comp' ~(Add ~x ~y) ~c &=& comp' ~x ~(comp' ~y ~(ADD : c)) \\
  \end{array}
\end{equation*}


%%% Local Variables: 
%%% mode: latex
%%% TeX-master: "master"
%%% End: 