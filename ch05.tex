\chapter{\Large{List comprehensions}}

%%%%%%%%%%%%%%%%%%%%%%%%%%%%%%%%%%%%%%%%%%%%%%%%%%%%%%%%%%%%%%%%%%%%%%
\exercise{5.1}
\begin{lstlisting}[language=Haskell]
Prelude> sum [x ^ 2 | x <- [1..100]]
338350
\end{lstlisting}

%%%%%%%%%%%%%%%%%%%%%%%%%%%%%%%%%%%%%%%%%%%%%%%%%%%%%%%%%%%%%%%%%%%%%%
\exercise{5.2}
\haskell{./src/ch05-ex02.hs}

%%%%%%%%%%%%%%%%%%%%%%%%%%%%%%%%%%%%%%%%%%%%%%%%%%%%%%%%%%%%%%%%%%%%%%
\exercise{5.3}
\haskell{./src/ch05-ex03.hs}

%%%%%%%%%%%%%%%%%%%%%%%%%%%%%%%%%%%%%%%%%%%%%%%%%%%%%%%%%%%%%%%%%%%%%%
\exercise{5.4}
\haskell[4]{./src/ch05-ex04.hs}

%%%%%%%%%%%%%%%%%%%%%%%%%%%%%%%%%%%%%%%%%%%%%%%%%%%%%%%%%%%%%%%%%%%%%%
\exercise{5.5}
\begin{lstlisting}[language=Haskell]
Prelude> concat [[(x, y) |  y <- [4, 5, 6]] | x <- [1, 2, 3]]
[(1,4),(1,5),(1,6),(2,4),(2,5),(2,6),(3,4),(3,5),(3,6)]
\end{lstlisting}

%%%%%%%%%%%%%%%%%%%%%%%%%%%%%%%%%%%%%%%%%%%%%%%%%%%%%%%%%%%%%%%%%%%%%%
\exercise{5.6}
\haskell[4]{./src/ch05-ex06.hs}

%%%%%%%%%%%%%%%%%%%%%%%%%%%%%%%%%%%%%%%%%%%%%%%%%%%%%%%%%%%%%%%%%%%%%%
\exercise{5.7}
\haskell{./src/ch05-ex07.hs}

%%%%%%%%%%%%%%%%%%%%%%%%%%%%%%%%%%%%%%%%%%%%%%%%%%%%%%%%%%%%%%%%%%%%%%
\exercise{5.8}

우선 다음과 같이 문자와 수 사이를 전환하는 `\texttt{int2let}'과
`\texttt{let2int}'를 대문자도 처리할 수 있도록 수정한다. 대문자는 알파벳
순으로 0에서 25사이의 수를 부여받고, 소문자는 그 다음 26에서 51사이의 수를
부여받는다.
\haskellpart{./src/ch05-ex08.hs}{16}{22}

가능한 문자가 총 52개이므로 이에 따라 `\texttt{shift}' 함수도 수정한다.
\haskellpart{./src/ch05-ex08.hs}{24}{27}

이제 기존의 `\texttt{encode}' 함수를 사용하여 대문자가 포함된 문장도 암호화할
수 있다. 다음은 대소문자가 포함된 문장을 암호화해 본 것이다.
\begin{lstlisting}[language=Haskell]
*Main> encode 5 "Hello my friend! Glad to see you."
"Mjqqt rD kwnjsi! Lqfi yt xjj Dtz."
\end{lstlisting}

이제 암호문을 푸는 프로그램이 대문자도 처리할 수 있도록 바꿔보자. 우선
대문자의 사용빈도 분포는 소문자와 같다고 하고 테이블을 수정한다.
\haskellpart{./src/ch05-ex08.hs}{31}{35}

주어진 문자열 내 사용빈도 구하는 함수 `\texttt{freqs}'를 수정하기 전에, 대문자
역시 고려 대상이므로 다음과 같이 대소문자의 수를 세는 함수를 작성한다.
\haskellpart{./src/ch05-ex08.hs}{10}{12}

함수 `\texttt{freqs}'가 대문자의 사용빈도도 구하도록 수정한다.
\haskellpart{./src/ch05-ex08.hs}{40}{42}

나머지는 이전과 동일하다. 다음은 작성한 프로그램을 실행해 본 모습이다.
\begin{lstlisting}[language=Haskell]
*Main> crack (encode 5 "Hello my friend! Glad to see you.")
"Hello my friend! Glad to see you."
\end{lstlisting}






%%% Local Variables: 
%%% mode: latex
%%% TeX-master: "master"
%%% End: 
